%einBlatt von Tempestrider & eisenherzz: Dieses Werk ist unter einem Creative Commons Attribution-NonCommercial-ShareAlike 3.0 Germany Lizenzvertrag lizenziert. Um die Lizenz anzusehen, gehen Sie bitte zu http://creativecommons.org/licenses/by-nc-sa/3.0/de/ oder schicken Sie einen Brief an Creative Commons, 171 Second Street, Suite 300, San Francisco, California 94105, USA.
\part {einRichtungen}
\chapter {einRichtungen}
einRichtungen sind Regelmodule, die
\begin{enumerate}
\item mit wenig Aufwand flexibel in Regelwerke, die auf einRegelBlatt basieren, eingebaut werden können
\item einen wichtigen Beitrag zur Beschreibung der Spielwelt oder der Charaktere leisten (folglich sind beispielsweise alle Regelmodule, die auf dem Charakterblatt Spuren hinterlassen, einRichtungen).
\end{enumerate}
einRichtungen sind zusammen mit einRegelBlatt die Bausteine, aus denen einBlatt-Regelwerke zusammengebaut werden.
Beispiele könnten einzelne Attribute sein (z.B. das Attribut Selbstbewusstsein), ebenso komplette Attributssysteme (wie die Standard Fantasy Attribute), Modifikationen am Attributs- oder Fertigkeitssystem (siehe Breite Fertigkeiten oder Freie Attribute), Erfahrungs-, Kampf-, Magie oder irgendwelche anderen (Sub-)Systeme, die eben fest mit den Charakteren (speziell der Charaktererschaffung) oder der Spielwelt verbunden sind.

%einBlatt von Tempestrider & eisenherzz: Dieses Werk ist unter einem Creative Commons Attribution-NonCommercial-ShareAlike 3.0 Germany Lizenzvertrag lizenziert. Um die Lizenz anzusehen, gehen Sie bitte zu http://creativecommons.org/licenses/by-nc-sa/3.0/de/ oder schicken Sie einen Brief an Creative Commons, 171 Second Street, Suite 300, San Francisco, California 94105, USA.
\section{Standard Punkteverteilung für Attribute}
\label{sect:StandardPunkteverteilungfuerAttribute}
Diese einRichtung liefert einen simplen Kaufmechanismus für die Ermittlung der Attributswerte. Anhand des gewünschten Machtlevels für die SCs bestimmt der SL eine Zahl von Punkten, die die Spieler eins-zu-eins in Attributspunkte tauschen können:
\begin{table}[H]
\caption{Standard Punkteverteilung für Attribute}
\begin{tabular}{|l|l|l|}
\hline
Punkteanzahl & Machtlevel & Beispiele \\
\hline
17 & Gentechnischischer Unfall & Homer S.; Jar Jar B.; Dieter B.;\\
21 & Normalo & Jay \& Silent Bob; Fuzzy; Du\\
24 & Spezialist & MacGyver; Das A-Team; Ich\\
27 & (Anti-)Held & Hartigan; Achilles; Der Joker\\
\hline
\end{tabular}
\end{table}
Der Mindestwert für ein Attribut liegt wie gehabt bei 2, der Maximalwert bei 10 Punkten. 
\\
\\
Diese Werte gehen davon aus, dass der SL in Proben etwa folgende Schwierigkeiten verwendet (bei manchen Herausforderungen sind mehrere mögliche Schwierigkeiten angegeben). SLs, die lieber höhere oder niedrigere Schwierigkeiten ansetzen, sollten die Punktezahl für die SCs entsprechend anpassen.
\begin{table}[H]
\caption{Schwierigkeiten}
\begin{tabular}{|p{10cm}|l|l|}
\hline
Herausforderung & Karten & Attributsfarben \\
\hline
eine einfache Kletterwand hinaufklettern (ohne Hilfsmittel) & 4 & keine\\
eine Felswand hinaufklettern (ohne Hilfsmittel) & 6 & eine\\
eine besonders schwierige Felswand hinaufklettern (ohne Hilfsmittel) & 	12 & eine\\
ein davonlaufendes Burgfräulein (im typischen Burgfräuleinkleid) einholen & 3 & eine\\
einen davonlaufenden, sehr athletischen Bösewicht einholen & 7 & eine\\
im typischen Burgfräuleinkleid vor einem sehr athletischen Bösewichtdavonlaufen & 7 & zwei\\
einen Dämon belügen & 12 & eine\\
\hline
\end{tabular}
\end{table}
Beispiel:
\\
\\
In einer ungerechten Galaxie wird Jar-Jar B. als dreizehntes und letztes Kind eines Alderaanischen Immobilienmaklers und einer halbintelligenten Waschmaschine geboren. Auf seine vier Attribute Intelligenz, Schnelligkeit, Kraft und Ausstrahlung verteilt er seine traurigen 17 Punkte wie folgt:
\begin{table}[H]
\caption{Die Attribute des Jar-Jar B.}
\begin{tabular}{|l|l|l|l|}
\hline
Kraft & Geschick & Intelligenz & Ausstrahlung \\
\hline
7 & 5 & 2 & 3\\
\hline
\end{tabular}
\end{table}
Zum Glück muss er niemals auf Atomphysikerkonferenzen große Vorträge halten, doch in einer der Nachwelt leider nicht erhalten gebliebenen Szene versucht er, Prinzessin Amidala (die gerade etwas unvorteilhaft gekleidet vor ihm flieht) einzuholen. Da Jar-Jar über die Geschick-Fertigkeit "`Schnell weg!"' verfügt, darf er fünf Karten verwenden (seine Attributsfarbe für Geschick -- Karo -- gilt ohnehin). Der SL legt eine Attributsfarbe für die fliehende Prinzessin fest und zieht drei Karten.

 
%einBlatt von Tempestrider & eisenherzz: Dieses Werk ist unter einem Creative Commons Attribution-NonCommercial-ShareAlike 3.0 Germany Lizenzvertrag lizenziert. Um die Lizenz anzusehen, gehen Sie bitte zu http://creativecommons.org/licenses/by-nc-sa/3.0/de/ oder schicken Sie einen Brief an Creative Commons, 171 Second Street, Suite 300, San Francisco, California 94105, USA.
\section{Breite Fertigkeiten}
Mit dieser einRichtung können SCs grundlegende Fertigkeiten in weiten Bereichen erhalten.
\subsubsection{Problem}
Nur in des seltensten Fällen wird ein SL eine Fertigkeit wie beispielsweise "`Bewaffneter Nahkampf"' erlauben -- sie ist einfach zu mächtig. Andererseits kann kein Krieger, selbst mit einem Attributswert von 10, alle Waffengattungen und Kampfstile als Fertigkeiten auflisten.

\subsubsection{In der Charaktererschaffung}
Um breite Grundfähigkeiten abzubilden, können diese ihrer Breite entsprechend mit einem Malus versehen werden, z.B.:
\begin{itemize}
\item Klingenwaffen: -2
\item Nahkampfwaffen: -3
\item Nahkampf: -4
\item Kämpfen: -5
\end{itemize}
Diese Fertigkeiten können mehrmals genommen werden, wodurch der Malus jeweils um 1 sinkt. Die Fertigkeit "`Kämpfen: -3"' würde also drei Fertigkeiten entsprechen (einmal um sie auf -5 zu nehmen und zwei weitere um sie auf -3 anzuheben). Unter -2 kann der Malus nicht gesenkt werden.

\subsubsection{In Proben}
Verwendet ein SC in einem Konflikt eine breite Fertigkeit, zieht er den Malus von der Anzahl der Karten, die er zieht, ab.



%einBlatt von Tempestrider & eisenherzz: Dieses Werk ist unter einem Creative Commons Attribution-NonCommercial-ShareAlike 3.0 Germany Lizenzvertrag lizenziert. Um die Lizenz anzusehen, gehen Sie bitte zu http://creativecommons.org/licenses/by-nc-sa/3.0/de/ oder schicken Sie einen Brief an Creative Commons, 171 Second Street, Suite 300, San Francisco, California 94105, USA.
\section{Erfahrungspunkte für Powergamer}
Erfahrungspunktesystem, dass sich nur an Konflikten und den in ihnen gespielten Karten orientiert.
\subsection{Erlangen von Erfahrungspunkten}

Die Spieler erhalten nach jeden Konflikt Erfahrungspunkte, wenn sie entweder :
\begin{itemize}
\item verloren
\item keine Bilder oder Asse gespielt
\item keine Attributsfarbe gespielt
\item gegen einen Gegner, der drei Bilder und / oder Asse gespielt hat, gewonnen haben.
\end{itemize}

Die Anzahl der Erfahrungspunkte ergibt sich aus folgender Aufstellung:

Der SC hat:
\begin{itemize}
\item Automatisch : 2 Punkte
\item Der Gegner war stärker: +3 Punkte
\item Verloren : + 1 Punkt
\item Pro Bild oder Ass, dass der Gegner gespielt hat: + 1 Punkt
\item Pro Bild oder Ass, dass der SC gespielt hat: - 2 Punkte
\end{itemize}

 

\subsection{Einsetzen von Erfahrungspunkten}

Eingesetzte Erfahrungspunkte sind für die Session verbraucht, werden aber zu Beginn der nächsten wieder "`regeneriert“'. Ihre Effekte sind jedoch meist ebenfalls nur von sehr vorübergehender Dauer.
\begin{itemize}
\item Die aktuelle Hand abwerfen und neu ziehen: 3 Punkte
\item Eine bereits ausgespielte gegnerische Karte abwerfen((Gegner spielt eine neue von seiner Hand): 4 Punkte
\item eine zusätzliche Karte ziehen: 1 Punkt
\item Für einen konflikt sämtliche Verletzungen ignorieren: 2 Punkte
\end{itemize}

 

\subsection{Ausgeben von Erfahrungspunkten}

Ausgegebene Erfahrungspunkte werden nicht regeneriert, ihr Effekt ist jedoch in aller Regel auch dauerhaft:
\begin{itemize}
\item Doch nicht tot sein: 3 Punkte
\item ein Attribut um einen Punkt steigern: aktueller Wert hoch 2 (von 2 auf 3: 4 Punkt, von 3 auf 4: 9 Punkte, von 5 auf 6: 25 Punkte, von 8 auf 9: 64 Punkte)
\item eine neue Fertigkeit lernen (wenn das Attribut bereits höher ist als die Anzahl der Fertigkeiten): Wert des Attributs
\item eine neue Fertigkeit lernen (wenn das Attribut nicht höher ist als die Anzahl der Fertigkeiten): Wert des Attributs * 3
\end{itemize}
 

Optional: Der SL kann eine Auswahl an einGriffen vorgeben, deren Nutzung Erfahrungspunkte kostet, beispielsweise:

 

%einBlatt von Tempestrider & eisenherzz: Dieses Werk ist unter einem Creative Commons Attribution-NonCommercial-ShareAlike 3.0 Germany Lizenzvertrag lizenziert. Um die Lizenz anzusehen, gehen Sie bitte zu http://creativecommons.org/licenses/by-nc-sa/3.0/de/ oder schicken Sie einen Brief an Creative Commons, 171 Second Street, Suite 300, San Francisco, California 94105, USA.
\section{Zusätzliche Eigenschaft: Selbstbewusstsein}
Eine Eigenschaft, die dem SC bei allen Würfen helfen kann, von ihm aber ein entsprechendes Verhalten verlangt.

\subsection{Charaktererschaffung:}

Selbstbewusstsein wird ergänzend zu einer beliebigen Attribute-Kombination eingesetzt. Bei der Punktezuteilung wird es normalerweise behandelt wie die anderen Attribute auch. Es hat keine eigene Attributsfarbe und anstelle von Fertigkeiten werden ihm Persönlichkeitsmerkmale (oder genauer: Facetten der Selbstwahrnehmung des SC) zugeordnet, aus denen sich dieses ergibt, z. B.:

\begin{itemize}
\item mutiger Kämpfer
\item treuer Freund
\item ein Mann der Tat
\item Überlebenskünstler
\item der Mann, den Mata Hari liebt
\item der Mann, den alle Frauen lieben
\item guter Verlierer
\item der, den alle ungerecht behandeln
\item \dots
\end{itemize}

\subsection{Zu Spielbeginn:}
Am Anfang jeder Session zieht der Spieler eine Kartenhand für sein Selbstbewusstsein, wirft die höchste und die niedrigste Karte daraus ab. Den Rest legt er verdeckt vor sich ab. Dieser Kartenvorrat ist nun seine Selbstbewusstseinshand.

\subsection{Selbstbewusstsein gewinnen:}
Nach jedem Konflikt, in dem es um sein Selbstbild geht (also eines der Persönlichkeitsmerkmale unter Selbstbewusstsein), kann er eine ungenutzte Karte aus diesem Konflikt auf seine Selbstbewusstseinshand legen. Wird seine Selbstbewusstseinshand damit größer als sein Selbstbewusstseinswert, muss er eine Karte von seiner Selbstbewusstseinshand abwerfen.

Auch wenn mehrere seiner Persönlichkeitsmerkmale betroffen sind, darf er pro Konflikt nur eine Karte auf die Selbstbewusstseinshand legen.

\subsection{Selbstbewusst handeln:}
Will ein SC in einem Konflikt auf sein Selbstbewusstsein zurückgreifen, so kann er zusätzlich zu seiner normalen Kartenhand für jedes Persönlichkeitsmerkmal, welches von diesem Konflikt betroffen ist, eine Karte aus seiner Selbstbewusstseinshand ausspielen.

Selbst wenn keines seiner Persönlichkeitsmerkmale betroffen ist, darf er eine Karte aus seiner Selbstbewusstseinshand spielen, wenn er
\begin{itemize}
\item freiwillig in den Konflikt gegangen ist und
\item der Konflikt ein erhebliches Risiko für ihn darstellt.
\end{itemize}
Es kann niemals in einem Konflikt gleichzeitig Selbstbewusstsein gewonnen und selbstbewusst gehandelt werden.

%einBlatt von Tempestrider & eisenherzz: Dieses Werk ist unter einem Creative Commons Attribution-NonCommercial-ShareAlike 3.0 Germany Lizenzvertrag lizenziert. Um die Lizenz anzusehen, gehen Sie bitte zu http://creativecommons.org/licenses/by-nc-sa/3.0/de/ oder schicken Sie einen Brief an Creative Commons, 171 Second Street, Suite 300, San Francisco, California 94105, USA.
\section{Anreizsystem (Attribute)}
\subsection{Zweck}
Diese Regelerweiterung soll die Spieler für ein vom SL oder vom Spiel bestimmtes Verhalten belohnen. Hierbei kann es sich z.B. um emotionales, heroisches oder einfach nur besonders kreatives Rollenspiel drehen.

\subsection{Regelanpassungen:}
\subsubsection{Allgemeine Regeln}
Alle "`7"' gelten als farblos, d.h. sie können nicht der Attributsfarbe angehören.
\subsubsection{Charaktererschaffung}
Das zu belohnende Verhalten muss im Vorfeld definiert und den Spielern bekannt gegeben werden. Im einfachsten Fall reicht die Bekanntmachung aus, dass beispielsweise jeder selbstlose Einsatz für die Hilflosen und Schwachen von den Göttern gesegnet wird.Um komplexere Effekte zu erzielen, bietet sich jedoch die Bindung an die Werte des Charakters an. Jedem Attribut wird (von den Regeln der einBildung, dem SL oder dem Spieler nach Vorgaben des SL) eine Motivation oder ein Ziel mitgegeben, z.B.:
\begin{itemize}
\item Attribut: Stärke\\
Motivation: Für den König\\
Bedeutung: "`Für meinen König bin ich doppelt so stark"'\\
\\
      oder
\item Attribut: Charisma\\
Motivation: Minne\\
Bedeutung: "`Erst im Werben um eine holde Maid kommt mein Charme auf Touren"'
\end{itemize}
\subsubsection{Spielvorbereitung}
Die Spieler entfernen sämtliche Asse, Könige und "`7"' aus ihrem Deck und legen diese beiseite. Faire SLs tun das gleiche :-)
\subsubsection{Im Spiel}
Handelt ein Charakter gegen eine Motivation, muss er eine (in besonders schweren Fällen auch zwei) Siebener auf seinen Ablagestapel legen (wodurch sie nach dem nächsten Mischen in den Nachziehstapel gelangen). Wenn ein Charakter sich einer Motivation entsprechend verhält, darf der Spieler den König der entsprechenden Farbe auf seinen Ablagestapel legen. Hat er sich den entsprechenden König bereits verdient, darf er stattdessen das Ass nehmen. Ist auch das nicht mehr verfügbar, darf er irgendeinen König oder wenn er auch die bereits alle hat ein beliebiges Ass wählen. Folgte der SC seiner Motivation sogar unter besonderer Opferbereitschaft oder großem Risiko, darf er Ass und König auf den Ablagestapel legen. Alternativ darf er eine "`7"' zurück zu den ausgesonderten Karten legen. Asse und Könige, die in einem Konflikt ausgespielt wurden, kehren zurück zu den aussortierten Karten. Sind sie aber bis zum Ende des Konflikts auf der Hand verblieben, werden sie auf den Ablagestapel gelegt und erneut in den nächsten Nachziehstapel eingemischt.




%einBlatt von Tempestrider & eisenherzz: Dieses Werk ist unter einem Creative Commons Attribution-NonCommercial-ShareAlike 3.0 Germany Lizenzvertrag lizenziert. Um die Lizenz anzusehen, gehen Sie bitte zu http://creativecommons.org/licenses/by-nc-sa/3.0/de/ oder schicken Sie einen Brief an Creative Commons, 171 Second Street, Suite 300, San Francisco, California 94105, USA.
\section{Lehren ziehen}
Zieht ein SC eine Lehre aus einer bedeutenden Szene (SL-Entscheidung), darf diese auf dem Charakterblatt vermerkt werden. Fortan darf er diese einmal pro Session in einem Konflikt beachten und eine zusätzliche Karte ziehen.


%einBlatt von Tempestrider & eisenherzz: Dieses Werk ist unter einem Creative Commons Attribution-NonCommercial-ShareAlike 3.0 Germany Lizenzvertrag lizenziert. Um die Lizenz anzusehen, gehen Sie bitte zu http://creativecommons.org/licenses/by-nc-sa/3.0/de/ oder schicken Sie einen Brief an Creative Commons, 171 Second Street, Suite 300, San Francisco, California 94105, USA.
\section{Zauberwörter Magiesystem}
\subsubsection{Voraussetzungen}
\begin{enumerate}
\item Es wird ein Attribut benötigt, von dem die magischen Fähigkeiten des SCs abhängig gemacht werden (z.B.: Weisheit, Mana, \dots). Da die Zauber über Fertigkeiten an diesem Attribut gebildet werden, müssen Magier sich zwischen ihren Zaubern und den weniger magischen Fertigkeiten entscheiden.
\item Falls gewünscht muss festgelegt werden, welchen Preis das Magietalent hat.
\end{enumerate}
\subsubsection{Charaktererschaffung}

Charaktere mit magischen Fähigkeiten verfügen unter dem Magieattribut statt Fertigkeiten über "`Zauberwörter"', die folgenden Kategoreien angehören:
\begin{itemize}
\item Prinzipien (Objekte, auf die ein Zauber wirken kann; Substantive)
\item Prozesse (Wirkungen von Zaubern; Verben)
\item Qualitäten (Modifikatoren für die Magie selbst; Adjektive)
\end{itemize}
All diese Zauberwörter werden weiter unten beschrieben.

Die Anzahl der Prinzipien und der Prozesse, über die ein SC verfügt, darf maximal um 2 auseinander liegen. Die der Qualitäten darf er frei bestimmen. Vom SL als "`mächtig"' eingestufte Zauberwörter müssen mit einem, "`besonders mächtige"' mit zwei Sternen markiert werden.

\subsection{Funktionsweise der Magie}

 
\subsubsection{Allgemeine Vorbemerkung}

Dieses Magiesystem ist sehr flexibel und interpretierbar. Einerseits bedeutet das, dass der Spieler einen großen kreativen Freiraum genießt. Andererseits bedeutet es aber auch, dass die Magie ihm nur selten völlig gehorchen wird -- sie ist eine eigene, freie Kraft, die genutzt, aber nicht berechnet oder gar beherrscht werden kann. In allem hat der SL das letzte Wort -- und ganz explizit das Recht, zu tun, was er will! Als Ratschlag sollte hier das "`Ja, aber \dots"' - Prinzip gelten: Der Zauber, den der Spieler angestrebt hat, wird gewirkt, aber je nach Ausgang der Zauberprobe passieren noch weitere unangenehmen Dinge oder gehen Teile schief.

\subsubsection{Prinzipien und Prozesse}

Im einfachsten Fall besteht der Effekt eines Zaubers immer aus einem Prozess (einer Handlung, ausgedrückt als Verb) und einem betroffenen Prinzip (ein Gegenstand im aller weitesten Sinne, ausgedrückt als Substantiv). So könnte man beispielsweise der allseits beliebte Flammenstrahl als "`Feuer bewegen"' beschrieben werden (mit dem Prinzip "`Feuer"' und dem Prozess "`bewegen"', ein Heilzauber als "`Fleisch stärken"' und ein Verhörzauber als "`Wahrheit erkennen"' -- oder, in einer weniger humanen Variante "`Lüge verbrennen"'. Die Einsatzmöglichkeiten von "`Dämonen rufen"' sollten selbsterklärend sein.
Unter Umständen kann ein Zauber auch zwei Prinzipien beinhalten, wobei das zweite als Ziel fungiert (es wird also dennoch genau ein Prozess benötigt). So könnte beispielsweise obiger Flammenstrahl um das Prinzip "`Feind"' zu "`Feuer auf Feind bewegen"' erweitert werden.

\subsubsection{Qualitäten}

Außerdem gibt es noch eine dritte Art von Zauberwörtern: Qualitäten, die das "`Wie"' eines Zaubers beschreiben. Mit Zustimmung des SL (die völlig willkürlich erteilt und widerrufen werden kann), dürfen Qualitäten auf Prinzipien oder Prozesse angewendet werden. So kann die Qualität "`vergangen"' Magie auf zurückliegende Ereignisse wirken lassen, "`geheim"' kann einen Zauber wie einen Zufall erscheinen lassen und nicht umsonst ist "`klebrig"' das erste Zauberwort aller kleinen Kobolde.
Ganz allgemein gilt, dass jedes Zauberwort vom SL abgesegnet werden muss.

\subsection{Wirkung und Grenzen von Magie}
\subsubsection{Reichweite}

Die meisten Zauberwörter wirken auf Kontakt. Dort, wo es offensichtlich sinnvoll ist gilt die natürliche Sichtweite als Reichweite. Auf der Zeitachse wirkt Magie jetzt und auf die Gegenwart.

\subsubsection{Wirkungsdauer}

Die Wirkungsdauer bestimmt in aller Willkür (aber bevor der Zauber gesprochen wird) der SL -- sie mag reichen von "`Zauberattribut in Sekunden"' über "`bis es jemand ändert"' bis hin zu "`für immer"'. Nur wenige Zauber haben jedoch eine große Lebensdauer.

\subsubsection{Grenzen verschieben}

Qualitäten wie "`künftig"', "`weit"' oder "`langlebig"' können Zauber jenseits dieser Grenzen ermöglichen. Hiervon wird jedoch im allgemeinen abgeraten -- zu groß sind die Risiken\dots
Ansonsten kann der SL Grenzverschiebungen anbieten und im Gegenzug die Zauberprobe schwerer machen -- hierzu ist er jedoch nicht verpflichtet.


\subsection{Die Gestaltung von Prozessen und Prinzipien}

\subsubsection{Breite contra enge Zauberwörter}

Die "`Breite eines Zauberworts"' bezeichnet die Größe seines Anwendungsbereichs. "`Wesen"' hat eine größere Breite als "`Mensch"', das wiederum breiter ist als "`Freund"' oder "`Feind"'. Die Breite hat folgende Bedeutung:
\begin{enumerate}
\item Ein breites Zauberwort wird der SC natürlich häufiger nutzen können
\item Andererseits kann er ein enges erheblich besser kontrollieren -- denn der Zauber ist stets nur genau so eng definiert, wie es seine Zauberworte tun. Eine Flammenlanze ("`Feuer bewegen"'), die um das Ziel "`Wesen"' erweitert wird, wird fast nie den nahenden schwarzen Ritter treffen -- zu viele Fliegen, Käfer und andere "`Wesen"' bieten sich als (aus Sicht der Magie) ebenbürtige Alternativen. Das Ziel "`Feind"' hingegen träfe wohl nur auf den schwarzen Ritter zu.
\item Den Nachteil breiter Worte können Qualitäten häufig lindern, wodurch aber der Zauber selbst natürlich wieder schwerer wird.
\end{enumerate}
 
\subsubsection{Keine "`flexiblen"' Zauberwörter}

Jeder, der auch nur einen Funken Powergamer in sich trägt, hat sich sicherlich bereits überlegt, welche Qualitäten aus breiten Prinzipien enge machen könnten. "`Bestimmt"' beispielsweise könnte doch aus eine Feuerlanze gegen Menschen eine Feuerlanze gegen "`bestimmte"' Menschen machen, oder?
Prinzipiell ja -- aber solange nicht ein Zauberwort diese Bestimmung vornimmt, tut es der SL allein, und wenn ein Zauberwort sagt, welche bestimmten Menschen, dann braucht man "`bestimmt"' nicht mehr. Es ist bedeutungslos und macht den Zauber nur schwerer.
Ähnlich sinnlos ist die Suche nach doppeldeutigen Zauberwörtern -- sie gelten immer nur in der ersten Bedeutung, in der sie verwendet wurden.

\subsubsection{Vorsilben vermeiden}

Besonders für Prozesse gilt, dass Vorsilben wenn irgend möglich zu vermeiden sind. So ist beispielsweise "`brennen"' ein sehr brauchbarer Prozess, "`verbrennen"' jedoch sollte der SL nicht zulassen. Der Zauber kann sein Ziel sicherlich in Flammen setzen -- ob es aber ganz und gar "`verbrennt"', hängt auch von vielen anderen Faktoren ab.
Wörter, bei denen der Sinn (und nicht nur das "`Boah!"' des Effekts) von der Vorsilbe abhängt, sind hiervon natürlich ausgenommen -- der Prozess "`wandeln"' kann das "`verwandeln"' nun mal nicht ersetzen\dots

\subsubsection{Schwarze Magie}

Seit Jahrhunderten schon werden die mächtigsten Zauberwörter unter Verschluss gehalten und ihr Gebrauch (den jeder Magier in weitem Umkreis sofort spürt) geächtet. Auf dem Codex der Unwörter stehen unter anderem:
\begin{itemize}
\item töten / sterben / leben / Tod / Leben / etc.
\item immer
\item ewig
\item überall
\item jeder / alle / viele / etc.
\item Zeit
\item Raum
\item Mensch / Person /
\item alles weitere, was der SL auf die Liste setzen möchte
\end{itemize}
Ebenso gilt gilt jeder zauber, der ein Lebewesen direkt tötet (und nicht nur eine tödliche Verletzung hervorruft) als schwarze Magie.

\subsection{Zauber wirken}

\subsubsection{Die Magieprobe}

Will ein SC einen Zauber wirken, so benennt er alle beteiligten Zauberwörter und einigt sich mit dem SL über den Effekt, den er damit erreichen möchte. Will der SC ein Prinzip einsetzen, über dass er nicht verfügt, so gilt dies als automatisch "`mächtig"' und natürlich als Einsatz einer nicht vorhandenen Fertigkeit (es darf also nur der halbe Attributswert verwendet werden). Pro Zauber darf maximal ein unbekanntes Wort verwendet werden. Wörter, die der SL als "`mächtig"' oder als "`besonders mächtig"' einstuft, dürfen nicht eingesetzt werden, ohne dass der SC sie beherrscht. Dann gibt der SL die Schwierigkeit bekannt. Diese ergibt sich wie folgt:
\begin{itemize}
\item jedes Zauberwort: +1
\item jedes als "`mächtig"' oder "`besonders mächtig"' eingestufte Zauberwort: zusätzlich +1
\item Der eigentliche Effekt des Zaubers: +1 (eine Feder schweben lassen) -- +5 (eine Stadtmauer sprengen).
\item Der SL kann dem Spieler anbieten, Wirkungsdauer, Reichweite etc. für eine höhere Schwierigkeit zu verändern
\end{itemize}

Die Attributsfarben, die der SL gegen den Zauber des SC wertet, ergeben sich wie folgt:
\begin{itemize}
\item Karo ist immer Attributsfarbe gegen den Magier
\item Herz wenn er schwarze Magie einsetzt
\item Kreuz wenn er mindestens zwei "`mächtige"' oder ein "`besonders mächtiges"' Zauberwort verwendet
\item Pik, wenn er mindestens zwei "`besonders mächtige"' Zauberworte verwendet
\end{itemize}
Erreicht der SC lediglich einen knappen Erfolg, wird wenigstens eines der Zauberworte eine "`unvorhergesehene"' Wirkung entfalten, der Zauber wird aber trotzdem keinen Schaden anrichten.\\
\\
Für einem klaren Erfolg bekommt der SC recht genau den angestrebten Effekt.\\
\\
Ein vollkommener Erfolg dehnt die Grenzen des Zaubers -- er trifft mehrere Ziele auf einmal, wirkt länger, stärker, etc., aber alles im Sinne des Magiers. 

In allen drei Fällen ist der Spieler selbst angehalten, Vorschläge für die genaue Wirkung zu machen.

\subsubsection{Die Widerstandsprobe}

Ist der Zauber gegen jemanden oder etwas gerichtet, legt der SL vor Zauberprobe fest, mit welchem Attribut und ggf. welcher Fertigkeit das Opfer versuchen darf, dem Zauber zu widerstehen. Um dies zu tun, tritt es als dritte Partei in die Zauberprobe ein.
Es sollten nur Fertigkeiten angewendet werden dürfen, die sehr genau passen -- oder eine spezifische Magieresistenz-Fertigkeit.

\subsubsection{Die Requisitenprobe}

Häufig wird ein Magier einen Teil eines Zaubers nicht mit seinen Zauberworten abdecken wollen, sondern weltlichere Lösung suchen. Hierfür muss eine Probe auf eine entsprechende nicht-magische Fertigkeit zusätzlich zu der Zauberprobe abgelegt werden. Viele Feuermagier erzeugen und bewegen ihre Feuerbälle magisch, zielen jedoch wie es jeder Bogenschütze tut.

\subsection{Beispiele}
\subsubsection{Voraussetzungen}

 
\subsubsection{Das Magie-Attribut}

Bei Verwendung der einRichtung "`Klassische Fantasy Attribute"'(siehe \ref {sec:todo}, Seite \pageref {sec:todo}) (Stärke, Geschick, Weisheit, Charisma) bietet sich natürlich Weisheit als Magie-Attribut an. Dies hat jedoch auch Nachteile: Unter dem Magieattribut werden Magier für gewöhnlich nur wenig anderes als ihre Magie-Fertigkeiten haben. Weisheit zum Magie-Attribut zu machen sorgt als für Magier, die außer vom Zaubern von nichts eine Ahnung haben -- Gandalf wäre in diesem System nicht abbildbar.
Eine Alternative besteht in Charisma: Dies lässt Magiern einerseits die Möglichkeit, fremde Sprachen zu lernen und Wissen über okkulte Traditionen zu sammeln, während es andererseits erklärt, warum fast alle Magier auf ihre Mitmenschen wie komplette Nerds wirken.

Welche Lösung hier die bessere ist (oder ob es in Deinem Setting besser passt, Magie an Stärke zu binden), muss allerdings immer von Fall zu Fall entschieden werden.
Der Preis des Magietalents

Es besteht eine Reihe verschiedener Möglichkeiten. Hier einige Beispiele:
\begin{itemize}
\item Keiner: Jeder Charakter darf sich magische Fähigkeiten nehmen
\item Attributsanforderungen: ein SC muss gewisse Mindestwerte in bestimmten Attributen erfüllen, um magische Fähigkeiten besitzen zu können
\item Attributsbeschränkungen: magiebegabte SCs dürfen in den körperlichen Attributen gewisse Maximalwerte nicht überschreiten
\item Magische Nachteile: für jede magische Fertigkeit muss der SCs einen magischen Nachteil erhalten, beispielsweise:
\begin{itemize} 
\item Der SC wird von bösen Feen als Spielzeug begehrt
\item Er ist ständig von Salpeterduft umgeben, außer eine hässliche Frau berührt ihn gerade (Charme - 1)
\item Er hat eintreibungswürdige Schulden bei der Magierakademie, die er Hals über Kopf verlassen hat, weil er nicht glaubte, die Prüfung bestehen zu können
\end{itemize}
\end{itemize}

\subsubsection{Funktionsweise der Magie}

\subsubsection{Prinzipien}
\begin{itemize} 
\item Feuer*, Wasser*, Erde*, Luft*
\item Lüge
\item Licht
\item Gedanke**
\item Kraft*
\item Weg*
\item Liebe**
\item Gier
\item Blut**
\item Schmerz
\end{itemize}

\subsubsection{Prozesse}
\begin{itemize} 
\item erkennen
\item bewegen
\item heilen
\item zerstören**
\item anhalten*
\item verschwinden*
\item verwandeln**
\item betören
\end{itemize}

\subsubsection{Qualitäten}
\begin{itemize} 
\item früher
\item schön
\item morgen
\item schnell
\item leise
\end{itemize}

\subsubsection{Wirkung und Grenzen von Magie}
\subsubsection{Reichweite}

Der Zauber "`Wunde heilen"' hat eine üblicherweise Reichweite von "`Berührung"'.
Der Zauber "`Feuer erschaffen bewegen"' (einen Feuerball erzeugen und schleudern) hat normalerweise eine Reichweite "`Sichtweite"'.
Der Zauber "`Lüge erkennen"' kann nur einen Betrug aufdecken, der im Moment des Zaubers oder wenige Minuten danach erfolgt. Um eine zurückliegende Aussage zu überprüfen, müsste der Magier eine Qualität wie "`vergangen"' einflechten -- und wenn er sich nicht am Ort der Aussage befindet oder den  Verdächtigen direkt mit dem Zauber belegen kann, benötigt er außerdem eine Qualität wie "`entfernt"'.

\subsubsection{Wirkungsdauer}

Hier sollte wirklich jeder SL seinen Weg finden.

\subsubsection{Grenzen verschieben}
Siehe Reichweite.

\subsubsection{Schwarze Magie}

Ein junger Adept hat im Labor seines Mentors ein paar Wunderlampen zu viel geputzt und soll nun zur Strafe das Abendessen für den ganzen Konvent richten. Doch schon beim Schlachten des Schweins verlässt ihn der Mut und er verlegt sich wieder auf eine Magische Lösung. Er berührt also das Schwein, wirkt seinen Zauber "`Blut** verschwinden"' und wird von einem wütenden, talarbekleideten Mob geächtet und davon gejagt, da sein Zauber verboten ist (da sein Zauber nur zu töten vermag). Zur gleichen Zeit entledigt sich sein leicht aufbrausender Mentor mittels eines Feuerballs einer Schnake. Auch er wird vom Hof gejagt, aber nur wegen der angesengten Tagebücher des Erzmagus -- ein Feuerball tötet nicht, er verwundet nur. Dass selbst ein hundert mal größeres Insekt den Feuerball nicht hätte überleben können spielt hierbei keine Rolle -- es geht nur um die prinzipielle Wirkungsweise des Zaubers.


\subsubsection{Zauber wirken}

\subsubsection{Die Magieprobe}

Ein SC will mit den Worten "`Feind"', "`Atem*"' und "`anhalten"' den Oberbösewicht aus der Welt schaffen. Leider isst der ein ganzes Stück weit weg und der SC muss das Wort "`entfernt"' improvisieren. So ergibt sich als Schwierigkeit:
\begin{itemize}
\item 4 Zauberwörter: +4
\item "`Atem*"' und das improvisierte "`entfernt"' = 2 "`mächtige"' Zauberwörter: +2
\item Der Effekt (nach Bewertung des SL): +4
\item Der SL zieht also 10 Karten -- nicht gerade ermutigend\dots
\end{itemize}
Nun zu den Attributsfarben:
\begin{itemize}
\item Karo ist immer Attributsfarbe gegen den Magier, also auch hier.
\item da er schwarze Magie einsetzt, gilt auch Herz als Attributsfarbe für den SL
\item Wegen seiner zwei "`mächtigen"' Zauberworte gilt auch Kreuz
\item Aber immerhin bleibt ihm Pik erspart, da er nicht mal ein "`besonders mächtiges"' Zauberwort verwendet.
\end{itemize}
Ich würde diesen Zauber nicht versuchen\dots

\subsubsection{Die Widerstandsprobe}

Da unser Test-Magier aber nicht so klug isst wie ich, legt der SL nun fest, dass sein Gegner mit seiner Gesundheit dagegen halten kann, also mit dem Attribut "`Stärke"'. Eine geeignete Fertigkeit wie "`Luft anhalten"' oder auch nur "`Tauchen"' hat das Opfer nicht, also kann es nur mit den halben Attributswert einsetzen.
Trotzdem verbessert das die ohnehin schlechten Aussichten unseres übernatürlichen Optimisten nicht gerade\dots

\subsubsection{Die Requisitenprobe}

Nachdem dieser kleine Mordanschlag gescheitert und ein neuer Magier-Charakter erschaffen ist, arbeitet sich die Gruppe auf traditionelle Weise zum Endgegner vor. Dort versucht sich der Neuling mit einem klassischen Feuerball:
"`Feuer"', "`erschaffen"', "`bewegen"'. Die Probe gelingt ihm vollkommen -- das wäre das Ende für den Bösewicht. Doch gehörte das Zielen nicht zum Zauber, also bestimmt der SL, dass eine erfolgreiche Wurfwaffenprobe gegen die "`Ausweichen"'-Fertigkeit des Erzfeindes über den Erfolg entscheidet.


%einBlatt von Tempestrider & eisenherzz: Dieses Werk ist unter einem Creative Commons Attribution-NonCommercial-ShareAlike 3.0 Germany Lizenzvertrag lizenziert. Um die Lizenz anzusehen, gehen Sie bitte zu http://creativecommons.org/licenses/by-nc-sa/3.0/de/ oder schicken Sie einen Brief an Creative Commons, 171 Second Street, Suite 300, San Francisco, California 94105, USA.
\section{einfache Fantasy-Attribute}
\label{sect:einfacheFantasyAttribute}
\subsection{Attribut: Stärke}

\subsubsection{Farbe: Kreuz}

\subsubsection{Bedeutung: Die körperliche Kraft des SC; seine Widerstandskraft; seine Gesundheit;}

\subsubsection{Fertigkeiten:}
\begin{itemize}
\item Heben
\item Standhalten (z.B. eine Tür geschlossen halten, gegen die auf der anderen Seite zehn Orks drücken).
\item Gesundheit (Widerstandskraft gegen Krankheiten, Wundentzündungen, etc.)
\item Schmerzen ertragen
\item Ausdauer
\item Trinkfestigkeit
\end{itemize}

\subsubsection{einSchränkungen:}

Alle Formen körperlicher Beeinträchtigungen, also z.B.:
\begin{itemize}
\item Wunden
\item Entzündungen
\item Zerrungen
\item Bänderrisse
\item Schnupfen
\item Pest
\item Cholera
\item Kopfweh (echtes)
\end{itemize}

\subsubsection{Erschöpfung:}

Sinkt Stärke auf 0, sollte man beten. Unter 0 ist es Zeit, ein Loch zu buddeln.

\subsection{Attribut: Gewandtheit}

\subsubsection{Farbe: Pik}

\subsubsection{Bedeutung: Schnelligkeit; Reflexe; Koordination; Nah- und Fernkampf}

\subsubsection{Fertigkeiten:}
\begin{itemize}
\item Schwerter
\item Äxte
\item Bogen
\item Ausweichen
\item Fingerfertigkeit
\item Akrobatik
\item Laufen
\item Schwimmen
\item Klettern
\end{itemize}

\subsubsection{einSchränkungen:}

Ähnlich wie bei Stärke, mit stärkerem Fokus auf Zerrungen, Dehnungen, Bänderrisse und weniger Wunden und Krankheiten (außer Rheuma, Gicht, etc.); aber auch ein Vollrausch kann das Geschick negativ beeinflussen.

\subsubsection{Erschöpfung:}

Fällt Gewandtheit auf 0, wird der SC unbeweglich und muss von anderen getragen werden.

\subsection{Attribut: Charisma}

\subsubsection{Farbe: Herz}

\subsubsection{Bedeutung: Selbstdarstellung; Empathie; Empfindsamkeit}

\subsubsection{Fertigkeiten:}
\begin{itemize}
\item Betören
\item Lügen
\item Umgang mit Tieren
\item Lügen erkennen
\item Agitation
\item Einschüchtern
\end{itemize}

\subsubsection{einSchränkungen:}

Alles, was die kontrollierte Wirkung auf andere oder die eigene Offenheit für andere einschränkt:
\begin{itemize}
\item eine häßliche Narbe
\item Pocken im Gesicht
\item übler Körpergeruch
\item unkontrolliertes Verhalten (z.B. durch einen Drogenrausch)
\item ein böses Gerücht
\end{itemize}

\subsubsection{Erschöpfung:}

Sinkt Charisma unter 3, wird der SC von wirklich ALLEN Menschen gemieden und aus jeder Gemeinschaft ausgeschlossen. Sein Selbstwertgefühl nimmt Schaden, was sich negativ auf JEDE seiner Handlungen auswirkt. Probt der SC in der Folge ein anderes Attribut als Charisma, muss er folgende Nachteile inkauf nehmen:
\begin{itemize}
\item bei Charisma 2 zieht er immer eine Karte weniger
\item bei Charisma 1 3 weniger
\item bei 0 und darunter darf er höchstens noch zwei Karten ziehen.
\end{itemize}

\subsection{Attribut: Weisheit}

\subsubsection{Farbe: Karo}

\subsubsection{Bedeutung: Wissen; Gedächtnis; Wahrnehmung; geistige Widerstandsfähigkeit; Willenskraft; Magie}

\subsubsection{Fertigkeiten:}
\begin{itemize}
\item Orientierung
\item Sagen \& Legenden
\item Etikette
\item Gedächtnis
\item Pflanzenkunde
\item Heilunde
\item Sehen
\item Hören
\item Wille
\item Rhetorik
\item Magieresistenz (wenn es das Magiesystem unterstützt)
\end{itemize}

\subsubsection{einSchränkungen:}

Alles, was die mentale Leistungsfähigkeit einschränkt: ein Vollrausch, ein Verwirrungszauber, ein fester Schlag auf den Kopf, \dots

\subsubsection{Erschöpfung:}

Fällt Weisheit unter 3, wird der SC sehr beeinflussbar und muss sich gegen Jeden Überredungsversuch mit seinem verbliebenen Willen verteidigen. Scheitert er, glaubt er das, was ihm gesagt wird, und handelt dementsprechend.

