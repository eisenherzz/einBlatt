%einBlatt von Tempestrider & eisenherzz: Dieses Werk ist unter einem Creative Commons Attribution-NonCommercial-ShareAlike 3.0 Germany Lizenzvertrag lizenziert. Um die Lizenz anzusehen, gehen Sie bitte zu http://creativecommons.org/licenses/by-nc-sa/3.0/de/ oder schicken Sie einen Brief an Creative Commons, 171 Second Street, Suite 300, San Francisco, California 94105, USA.
\part {einRichtungen}
\chapter {einRichtungen}
einRichtungen sind Regelmodule, die
\begin{enumerate}
\item mit wenig Aufwand flexibel in Regelwerke, die auf einRegelBlatt basieren, eingebaut werden können
\item einen wichtigen Beitrag zur Beschreibung der Spielwelt oder der Charaktere leisten (folglich sind beispielsweise alle Regelmodule, die auf dem Charakterblatt Spuren hinterlassen, einRichtungen).
\end{enumerate}
einRichtungen sind zusammen mit einRegelBlatt die Bausteine, aus denen einBlatt-Regelwerke zusammengebaut werden.
Beispiele könnten einzelne Attribute sein (z.B. das Attribut Selbstbewusstsein), ebenso komplette Attributssysteme (wie die Standard Fantasy Attribute), Modifikationen am Attributs- oder Fertigkeitssystem (siehe Breite Fertigkeiten oder Freie Attribute), Erfahrungs-, Kampf-, Magie oder irgendwelche anderen (Sub-)Systeme, die eben fest mit den Charakteren (speziell der Charaktererschaffung) oder der Spielwelt verbunden sind.

%einBlatt von Tempestrider & eisenherzz: Dieses Werk ist unter einem Creative Commons Attribution-NonCommercial-ShareAlike 3.0 Germany Lizenzvertrag lizenziert. Um die Lizenz anzusehen, gehen Sie bitte zu http://creativecommons.org/licenses/by-nc-sa/3.0/de/ oder schicken Sie einen Brief an Creative Commons, 171 Second Street, Suite 300, San Francisco, California 94105, USA.
\section{Breite Fertigkeiten}
Mit dieser einRichtung können SCs grundlegende Fertigkeiten in weiten Bereichen erhalten.
\subsubsection{Problem}
Nur in des seltensten Fällen wird ein SL eine Fertigkeit wie beispielsweise "`Bewaffneter Nahkampf"' erlauben -- sie ist einfach zu mächtig. Andererseits kann kein Krieger, selbst mit einem Attributswert von 10, alle Waffengattungen und Kampfstile als Fertigkeiten auflisten.

\subsubsection{In der Charaktererschaffung}
Um breite Grundfähigkeiten abzubilden, können diese ihrer Breite entsprechend mit einem Malus versehen werden, z.B.:
\begin{itemize}
\item Klingenwaffen: -2
\item Nahkampfwaffen: -3
\item Nahkampf: -4
\item Kämpfen: -5
\end{itemize}
Diese Fertigkeiten können mehrmals genommen werden, wodurch der Malus jeweils um 1 sinkt. Die Fertigkeit "`Kämpfen: -3"' würde also drei Fertigkeiten entsprechen (einmal um sie auf -5 zu nehmen und zwei weitere um sie auf -3 anzuheben). Unter -2 kann der Malus nicht gesenkt werden.

\subsubsection{In Proben}
Verwendet ein SC in einem Konflikt eine breite Fertigkeit, zieht er den Malus von der Anzahl der Karten, die er zieht, ab.



