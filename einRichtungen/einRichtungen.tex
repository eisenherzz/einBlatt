%einBlatt von Tempestrider & eisenherzz: Dieses Werk ist unter einem Creative Commons Attribution-NonCommercial-ShareAlike 3.0 Germany Lizenzvertrag lizenziert. Um die Lizenz anzusehen, gehen Sie bitte zu http://creativecommons.org/licenses/by-nc-sa/3.0/de/ oder schicken Sie einen Brief an Creative Commons, 171 Second Street, Suite 300, San Francisco, California 94105, USA.
\part {einRichtungen}
\chapter {einRichtungen}
einRichtungen sind Regelmodule, die
\begin{enumerate}
\item mit wenig Aufwand flexibel in Regelwerke, die auf einRegelBlatt basieren, eingebaut werden können
\item einen wichtigen Beitrag zur Beschreibung der Spielwelt oder der Charaktere leisten (folglich sind beispielsweise alle Regelmodule, die auf dem Charakterblatt Spuren hinterlassen, einRichtungen).
\end{enumerate}
einRichtungen sind zusammen mit einRegelBlatt die Bausteine, aus denen einBlatt-Regelwerke zusammengebaut werden.
Beispiele könnten einzelne Attribute sein (z.B. das Attribut Selbstbewusstsein), ebenso komplette Attributssysteme (wie die Standard Fantasy Attribute), Modifikationen am Attributs- oder Fertigkeitssystem (siehe Breite Fertigkeiten oder Freie Attribute), Erfahrungs-, Kampf-, Magie oder irgendwelche anderen (Sub-)Systeme, die eben fest mit den Charakteren (speziell der Charaktererschaffung) oder der Spielwelt verbunden sind.

%einBlatt von Tempestrider & eisenherzz: Dieses Werk ist unter einem Creative Commons Attribution-NonCommercial-ShareAlike 3.0 Germany Lizenzvertrag lizenziert. Um die Lizenz anzusehen, gehen Sie bitte zu http://creativecommons.org/licenses/by-nc-sa/3.0/de/ oder schicken Sie einen Brief an Creative Commons, 171 Second Street, Suite 300, San Francisco, California 94105, USA.
\section{Standard Punkteverteilung für Attribute}
\label{sect:StandardPunkteverteilungfuerAttribute}
Diese einRichtung liefert einen simplen Kaufmechanismus für die Ermittlung der Attributswerte. Anhand des gewünschten Machtlevels für die SCs bestimmt der SL eine Zahl von Punkten, die die Spieler eins-zu-eins in Attributspunkte tauschen können:
\begin{table}[H]
\caption{Standard Punkteverteilung für Attribute}
\begin{tabular}{|l|l|l|}
\hline
Punkteanzahl & Machtlevel & Beispiele \\
\hline
17 & Gentechnischischer Unfall & Homer S.; Jar Jar B.; Dieter B.;\\
21 & Normalo & Jay \& Silent Bob; Fuzzy; Du\\
24 & Spezialist & MacGyver; Das A-Team; Ich\\
27 & (Anti-)Held & Hartigan; Achilles; Der Joker\\
\hline
\end{tabular}
\end{table}
Der Mindestwert für ein Attribut liegt wie gehabt bei 2, der Maximalwert bei 10 Punkten. 
\\
\\
Diese Werte gehen davon aus, dass der SL in Proben etwa folgende Schwierigkeiten verwendet (bei manchen Herausforderungen sind mehrere mögliche Schwierigkeiten angegeben). SLs, die lieber höhere oder niedrigere Schwierigkeiten ansetzen, sollten die Punktezahl für die SCs entsprechend anpassen.
\begin{table}[H]
\caption{Schwierigkeiten}
\begin{tabular}{|p{10cm}|l|l|}
\hline
Herausforderung & Karten & Attributsfarben \\
\hline
eine einfache Kletterwand hinaufklettern (ohne Hilfsmittel) & 4 & keine\\
eine Felswand hinaufklettern (ohne Hilfsmittel) & 6 & eine\\
eine besonders schwierige Felswand hinaufklettern (ohne Hilfsmittel) & 	12 & eine\\
ein davonlaufendes Burgfräulein (im typischen Burgfräuleinkleid) einholen & 3 & eine\\
einen davonlaufenden, sehr athletischen Bösewicht einholen & 7 & eine\\
im typischen Burgfräuleinkleid vor einem sehr athletischen Bösewichtdavonlaufen & 7 & zwei\\
einen Dämon belügen & 12 & eine\\
\hline
\end{tabular}
\end{table}
Beispiel:
\\
\\
In einer ungerechten Galaxie wird Jar-Jar B. als dreizehntes und letztes Kind eines Alderaanischen Immobilienmaklers und einer halbintelligenten Waschmaschine geboren. Auf seine vier Attribute Intelligenz, Schnelligkeit, Kraft und Ausstrahlung verteilt er seine traurigen 17 Punkte wie folgt:
\begin{table}[H]
\caption{Die Attribute des Jar-Jar B.}
\begin{tabular}{|l|l|l|l|}
\hline
Kraft & Geschick & Intelligenz & Ausstrahlung \\
\hline
7 & 5 & 2 & 3\\
\hline
\end{tabular}
\end{table}
Zum Glück muss er niemals auf Atomphysikerkonferenzen große Vorträge halten, doch in einer der Nachwelt leider nicht erhalten gebliebenen Szene versucht er, Prinzessin Amidala (die gerade etwas unvorteilhaft gekleidet vor ihm flieht) einzuholen. Da Jar-Jar über die Geschick-Fertigkeit "`Schnell weg!"' verfügt, darf er fünf Karten verwenden (seine Attributsfarbe für Geschick -- Karo -- gilt ohnehin). Der SL legt eine Attributsfarbe für die fliehende Prinzessin fest und zieht drei Karten.

 
%einBlatt von Tempestrider & eisenherzz: Dieses Werk ist unter einem Creative Commons Attribution-NonCommercial-ShareAlike 3.0 Germany Lizenzvertrag lizenziert. Um die Lizenz anzusehen, gehen Sie bitte zu http://creativecommons.org/licenses/by-nc-sa/3.0/de/ oder schicken Sie einen Brief an Creative Commons, 171 Second Street, Suite 300, San Francisco, California 94105, USA.
\section{Breite Fertigkeiten}
Mit dieser einRichtung können SCs grundlegende Fertigkeiten in weiten Bereichen erhalten.
\subsubsection{Problem}
Nur in des seltensten Fällen wird ein SL eine Fertigkeit wie beispielsweise "`Bewaffneter Nahkampf"' erlauben -- sie ist einfach zu mächtig. Andererseits kann kein Krieger, selbst mit einem Attributswert von 10, alle Waffengattungen und Kampfstile als Fertigkeiten auflisten.

\subsubsection{In der Charaktererschaffung}
Um breite Grundfähigkeiten abzubilden, können diese ihrer Breite entsprechend mit einem Malus versehen werden, z.B.:
\begin{itemize}
\item Klingenwaffen: -2
\item Nahkampfwaffen: -3
\item Nahkampf: -4
\item Kämpfen: -5
\end{itemize}
Diese Fertigkeiten können mehrmals genommen werden, wodurch der Malus jeweils um 1 sinkt. Die Fertigkeit "`Kämpfen: -3"' würde also drei Fertigkeiten entsprechen (einmal um sie auf -5 zu nehmen und zwei weitere um sie auf -3 anzuheben). Unter -2 kann der Malus nicht gesenkt werden.

\subsubsection{In Proben}
Verwendet ein SC in einem Konflikt eine breite Fertigkeit, zieht er den Malus von der Anzahl der Karten, die er zieht, ab.



%einBlatt von Tempestrider & eisenherzz: Dieses Werk ist unter einem Creative Commons Attribution-NonCommercial-ShareAlike 3.0 Germany Lizenzvertrag lizenziert. Um die Lizenz anzusehen, gehen Sie bitte zu http://creativecommons.org/licenses/by-nc-sa/3.0/de/ oder schicken Sie einen Brief an Creative Commons, 171 Second Street, Suite 300, San Francisco, California 94105, USA.
\section{Erfahrungspunkte für Powergamer}
Erfahrungspunktesystem, dass sich nur an Konflikten und den in ihnen gespielten Karten orientiert.
\subsection{Erlangen von Erfahrungspunkten}

Die Spieler erhalten nach jeden Konflikt Erfahrungspunkte, wenn sie entweder :
\begin{itemize}
\item verloren
\item keine Bilder oder Asse gespielt
\item keine Attributsfarbe gespielt
\item gegen einen Gegner, der drei Bilder und / oder Asse gespielt hat, gewonnen haben.
\end{itemize}

Die Anzahl der Erfahrungspunkte ergibt sich aus folgender Aufstellung:

Der SC hat:
\begin{itemize}
\item Automatisch : 2 Punkte
\item Der Gegner war stärker: +3 Punkte
\item Verloren : + 1 Punkt
\item Pro Bild oder Ass, dass der Gegner gespielt hat: + 1 Punkt
\item Pro Bild oder Ass, dass der SC gespielt hat: - 2 Punkte
\end{itemize}

 

\subsection{Einsetzen von Erfahrungspunkten}

Eingesetzte Erfahrungspunkte sind für die Session verbraucht, werden aber zu Beginn der nächsten wieder "`regeneriert“'. Ihre Effekte sind jedoch meist ebenfalls nur von sehr vorübergehender Dauer.
\begin{itemize}
\item Die aktuelle Hand abwerfen und neu ziehen: 3 Punkte
\item Eine bereits ausgespielte gegnerische Karte abwerfen((Gegner spielt eine neue von seiner Hand): 4 Punkte
\item eine zusätzliche Karte ziehen: 1 Punkt
\item Für einen konflikt sämtliche Verletzungen ignorieren: 2 Punkte
\end{itemize}

 

\subsection{Ausgeben von Erfahrungspunkten}

Ausgegebene Erfahrungspunkte werden nicht regeneriert, ihr Effekt ist jedoch in aller Regel auch dauerhaft:
\begin{itemize}
\item Doch nicht tot sein: 3 Punkte
\item ein Attribut um einen Punkt steigern: aktueller Wert hoch 2 (von 2 auf 3: 4 Punkt, von 3 auf 4: 9 Punkte, von 5 auf 6: 25 Punkte, von 8 auf 9: 64 Punkte)
\item eine neue Fertigkeit lernen (wenn das Attribut bereits höher ist als die Anzahl der Fertigkeiten): Wert des Attributs
\item eine neue Fertigkeit lernen (wenn das Attribut nicht höher ist als die Anzahl der Fertigkeiten): Wert des Attributs * 3
\end{itemize}
 

Optional: Der SL kann eine Auswahl an einGriffen vorgeben, deren Nutzung Erfahrungspunkte kostet, beispielsweise:

 

%einBlatt von Tempestrider & eisenherzz: Dieses Werk ist unter einem Creative Commons Attribution-NonCommercial-ShareAlike 3.0 Germany Lizenzvertrag lizenziert. Um die Lizenz anzusehen, gehen Sie bitte zu http://creativecommons.org/licenses/by-nc-sa/3.0/de/ oder schicken Sie einen Brief an Creative Commons, 171 Second Street, Suite 300, San Francisco, California 94105, USA.
\section{Zusätzliche Eigenschaft: Selbstbewusstsein}
Eine Eigenschaft, die dem SC bei allen Würfen helfen kann, von ihm aber ein entsprechendes Verhalten verlangt.

\subsection{Charaktererschaffung:}

Selbstbewusstsein wird ergänzend zu einer beliebigen Attribute-Kombination eingesetzt. Bei der Punktezuteilung wird es normalerweise behandelt wie die anderen Attribute auch. Es hat keine eigene Attributsfarbe und anstelle von Fertigkeiten werden ihm Persönlichkeitsmerkmale (oder genauer: Facetten der Selbstwahrnehmung des SC) zugeordnet, aus denen sich dieses ergibt, z. B.:

\begin{itemize}
\item mutiger Kämpfer
\item treuer Freund
\item ein Mann der Tat
\item Überlebenskünstler
\item der Mann, den Mata Hari liebt
\item der Mann, den alle Frauen lieben
\item guter Verlierer
\item der, den alle ungerecht behandeln
\item \dots
\end{itemize}

\subsection{Zu Spielbeginn:}
Am Anfang jeder Session zieht der Spieler eine Kartenhand für sein Selbstbewusstsein, wirft die höchste und die niedrigste Karte daraus ab. Den Rest legt er verdeckt vor sich ab. Dieser Kartenvorrat ist nun seine Selbstbewusstseinshand.

\subsection{Selbstbewusstsein gewinnen:}
Nach jedem Konflikt, in dem es um sein Selbstbild geht (also eines der Persönlichkeitsmerkmale unter Selbstbewusstsein), kann er eine ungenutzte Karte aus diesem Konflikt auf seine Selbstbewusstseinshand legen. Wird seine Selbstbewusstseinshand damit größer als sein Selbstbewusstseinswert, muss er eine Karte von seiner Selbstbewusstseinshand abwerfen.

Auch wenn mehrere seiner Persönlichkeitsmerkmale betroffen sind, darf er pro Konflikt nur eine Karte auf die Selbstbewusstseinshand legen.

\subsection{Selbstbewusst handeln:}
Will ein SC in einem Konflikt auf sein Selbstbewusstsein zurückgreifen, so kann er zusätzlich zu seiner normalen Kartenhand für jedes Persönlichkeitsmerkmal, welches von diesem Konflikt betroffen ist, eine Karte aus seiner Selbstbewusstseinshand ausspielen.

Selbst wenn keines seiner Persönlichkeitsmerkmale betroffen ist, darf er eine Karte aus seiner Selbstbewusstseinshand spielen, wenn er
\begin{itemize}
\item freiwillig in den Konflikt gegangen ist und
\item der Konflikt ein erhebliches Risiko für ihn darstellt.
\end{itemize}
Es kann niemals in einem Konflikt gleichzeitig Selbstbewusstsein gewonnen und selbstbewusst gehandelt werden.

