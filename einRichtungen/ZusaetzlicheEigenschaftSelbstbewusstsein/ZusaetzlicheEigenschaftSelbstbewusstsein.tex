%einBlatt von Tempestrider & eisenherzz: Dieses Werk ist unter einem Creative Commons Attribution-NonCommercial-ShareAlike 3.0 Germany Lizenzvertrag lizenziert. Um die Lizenz anzusehen, gehen Sie bitte zu http://creativecommons.org/licenses/by-nc-sa/3.0/de/ oder schicken Sie einen Brief an Creative Commons, 171 Second Street, Suite 300, San Francisco, California 94105, USA.
\section{Zusätzliche Eigenschaft: Selbstbewusstsein}
Eine Eigenschaft, die dem SC bei allen Würfen helfen kann, von ihm aber ein entsprechendes Verhalten verlangt.

\subsubsection{Charaktererschaffung}

Selbstbewusstsein wird ergänzend zu einer beliebigen Attribute-Kombination eingesetzt. Bei der Punktezuteilung wird es normalerweise behandelt wie die anderen Attribute auch. Es hat keine eigene Attributsfarbe und anstelle von Fertigkeiten werden ihm Persönlichkeitsmerkmale (oder genauer: Facetten der Selbstwahrnehmung des SC) zugeordnet, aus denen sich dieses ergibt, z. B.:

\begin{itemize}
\item mutiger Kämpfer
\item treuer Freund
\item ein Mann der Tat
\item Überlebenskünstler
\item der Mann, den Mata Hari liebt
\item der Mann, den alle Frauen lieben
\item guter Verlierer
\item der, den alle ungerecht behandeln
\item \dots
\end{itemize}

\subsubsection{Zu Spielbeginn}
Am Anfang jeder Session zieht der Spieler eine Kartenhand für sein Selbstbewusstsein, wirft die höchste und die niedrigste Karte daraus ab. Den Rest legt er verdeckt vor sich ab. Dieser Kartenvorrat ist nun seine Selbstbewusstseinshand.

\subsubsection{Selbstbewusstsein gewinnen}
Nach jedem Konflikt, in dem es um sein Selbstbild geht (also eines der Persönlichkeitsmerkmale unter Selbstbewusstsein), kann er eine ungenutzte Karte aus diesem Konflikt auf seine Selbstbewusstseinshand legen. Wird seine Selbstbewusstseinshand damit größer als sein Selbstbewusstseinswert, muss er eine Karte von seiner Selbstbewusstseinshand abwerfen.

Auch wenn mehrere seiner Persönlichkeitsmerkmale betroffen sind, darf er pro Konflikt nur eine Karte auf die Selbstbewusstseinshand legen.

\subsubsection{Selbstbewusst handeln}
Will ein SC in einem Konflikt auf sein Selbstbewusstsein zurückgreifen, so kann er zusätzlich zu seiner normalen Kartenhand für jedes Persönlichkeitsmerkmal, welches von diesem Konflikt betroffen ist, eine Karte aus seiner Selbstbewusstseinshand ausspielen.

Selbst wenn keines seiner Persönlichkeitsmerkmale betroffen ist, darf er eine Karte aus seiner Selbstbewusstseinshand spielen, wenn er
\begin{itemize}
\item freiwillig in den Konflikt gegangen ist und
\item der Konflikt ein erhebliches Risiko für ihn darstellt.
\end{itemize}
Es kann niemals in einem Konflikt gleichzeitig Selbstbewusstsein gewonnen und selbstbewusst gehandelt werden.
