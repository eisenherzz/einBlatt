%einBlatt von Tempestrider & eisenherzz: Dieses Werk ist unter einem Creative Commons Attribution-NonCommercial-ShareAlike 3.0 Germany Lizenzvertrag lizenziert. Um die Lizenz anzusehen, gehen Sie bitte zu http://creativecommons.org/licenses/by-nc-sa/3.0/de/ oder schicken Sie einen Brief an Creative Commons, 171 Second Street, Suite 300, San Francisco, California 94105, USA.
\section{Erfahrungspunkte für Powergamer}
Erfahrungspunktesystem, dass sich nur an Konflikten und den in ihnen gespielten Karten orientiert.
\subsection{Erlangen von Erfahrungspunkten}

Die Spieler erhalten nach jeden Konflikt Erfahrungspunkte, wenn sie entweder :
\begin{itemize}
\item verloren
\item keine Bilder oder Asse gespielt
\item keine Attributsfarbe gespielt
\item gegen einen Gegner, der drei Bilder und / oder Asse gespielt hat, gewonnen haben.
\end{itemize}

Die Anzahl der Erfahrungspunkte ergibt sich aus folgender Aufstellung:

Der SC hat:
\begin{itemize}
\item Automatisch : 2 Punkte
\item Der Gegner war stärker: +3 Punkte
\item Verloren : + 1 Punkt
\item Pro Bild oder Ass, dass der Gegner gespielt hat: + 1 Punkt
\item Pro Bild oder Ass, dass der SC gespielt hat: - 2 Punkte
\end{itemize}

 

\subsection{Einsetzen von Erfahrungspunkten}

Eingesetzte Erfahrungspunkte sind für die Session verbraucht, werden aber zu Beginn der nächsten wieder "`regeneriert“'. Ihre Effekte sind jedoch meist ebenfalls nur von sehr vorübergehender Dauer.
\begin{itemize}
\item Die aktuelle Hand abwerfen und neu ziehen: 3 Punkte
\item Eine bereits ausgespielte gegnerische Karte abwerfen((Gegner spielt eine neue von seiner Hand): 4 Punkte
\item eine zusätzliche Karte ziehen: 1 Punkt
\item Für einen konflikt sämtliche Verletzungen ignorieren: 2 Punkte
\end{itemize}

 

\subsection{Ausgeben von Erfahrungspunkten}

Ausgegebene Erfahrungspunkte werden nicht regeneriert, ihr Effekt ist jedoch in aller Regel auch dauerhaft:
\begin{itemize}
\item Doch nicht tot sein: 3 Punkte
\item ein Attribut um einen Punkt steigern: aktueller Wert hoch 2 (von 2 auf 3: 4 Punkt, von 3 auf 4: 9 Punkte, von 5 auf 6: 25 Punkte, von 8 auf 9: 64 Punkte)
\item eine neue Fertigkeit lernen (wenn das Attribut bereits höher ist als die Anzahl der Fertigkeiten): Wert des Attributs
\item eine neue Fertigkeit lernen (wenn das Attribut nicht höher ist als die Anzahl der Fertigkeiten): Wert des Attributs * 3
\end{itemize}
 

Optional: Der SL kann eine Auswahl an einGriffen vorgeben, deren Nutzung Erfahrungspunkte kostet, beispielsweise:

 
