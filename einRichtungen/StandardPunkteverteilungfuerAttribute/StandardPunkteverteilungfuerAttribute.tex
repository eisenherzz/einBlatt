%einBlatt von Tempestrider & eisenherzz: Dieses Werk ist unter einem Creative Commons Attribution-NonCommercial-ShareAlike 3.0 Germany Lizenzvertrag lizenziert. Um die Lizenz anzusehen, gehen Sie bitte zu http://creativecommons.org/licenses/by-nc-sa/3.0/de/ oder schicken Sie einen Brief an Creative Commons, 171 Second Street, Suite 300, San Francisco, California 94105, USA.
\section{Standard Punkteverteilung für Attribute}
\label{sect:StandardPunkteverteilungfuerAttribute}
Diese einRichtung liefert einen simplen Kaufmechanismus für die Ermittlung der Attributswerte. Anhand des gewünschten Machtlevels für die SCs bestimmt der SL eine Zahl von Punkten, die die Spieler eins-zu-eins in Attributspunkte tauschen können:
\begin{table}[H]
\caption{Standard Punkteverteilung für Attribute}
\begin{tabular}{|l|l|l|}
\hline
Punkteanzahl & Machtlevel & Beispiele \\
\hline
17 & Gentechnischischer Unfall & Homer S.; Jar Jar B.; Dieter B.;\\
21 & Normalo & Jay \& Silent Bob; Fuzzy; Du\\
24 & Spezialist & MacGyver; Das A-Team; Ich\\
27 & (Anti-)Held & Hartigan; Achilles; Der Joker\\
\hline
\end{tabular}
\end{table}
Der Mindestwert für ein Attribut liegt wie gehabt bei 2, der Maximalwert bei 10 Punkten. 
\\
\\
Diese Werte gehen davon aus, dass der SL in Proben etwa folgende Schwierigkeiten verwendet (bei manchen Herausforderungen sind mehrere mögliche Schwierigkeiten angegeben). SLs, die lieber höhere oder niedrigere Schwierigkeiten ansetzen, sollten die Punktezahl für die SCs entsprechend anpassen.
\begin{table}[H]
\caption{Schwierigkeiten}
\begin{tabular}{|p{10cm}|l|l|}
\hline
Herausforderung & Karten & Attributsfarben \\
\hline
eine einfache Kletterwand hinaufklettern (ohne Hilfsmittel) & 4 & keine\\
eine Felswand hinaufklettern (ohne Hilfsmittel) & 6 & eine\\
eine besonders schwierige Felswand hinaufklettern (ohne Hilfsmittel) & 	12 & eine\\
ein davonlaufendes Burgfräulein (im typischen Burgfräuleinkleid) einholen & 3 & eine\\
einen davonlaufenden, sehr athletischen Bösewicht einholen & 7 & eine\\
im typischen Burgfräuleinkleid vor einem sehr athletischen Bösewichtdavonlaufen & 7 & zwei\\
einen Dämon belügen & 12 & eine\\
\hline
\end{tabular}
\end{table}
Beispiel:
\\
\\
In einer ungerechten Galaxie wird Jar-Jar B. als dreizehntes und letztes Kind eines Alderaanischen Immobilienmaklers und einer halbintelligenten Waschmaschine geboren. Auf seine vier Attribute Intelligenz, Schnelligkeit, Kraft und Ausstrahlung verteilt er seine traurigen 17 Punkte wie folgt:
\begin{table}[H]
\caption{Die Attribute des Jar-Jar B.}
\begin{tabular}{|l|l|l|l|}
\hline
Kraft & Geschick & Intelligenz & Ausstrahlung \\
\hline
7 & 5 & 2 & 3\\
\hline
\end{tabular}
\end{table}
Zum Glück muss er niemals auf Atomphysikerkonferenzen große Vorträge halten, doch in einer der Nachwelt leider nicht erhalten gebliebenen Szene versucht er, Prinzessin Amidala (die gerade etwas unvorteilhaft gekleidet vor ihm flieht) einzuholen. Da Jar-Jar über die Geschick-Fertigkeit "`Schnell weg!"' verfügt, darf er fünf Karten verwenden (seine Attributsfarbe für Geschick -- Karo -- gilt ohnehin). Der SL legt eine Attributsfarbe für die fliehende Prinzessin fest und zieht drei Karten.

