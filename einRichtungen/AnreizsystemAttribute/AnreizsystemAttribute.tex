%einBlatt von Tempestrider & eisenherzz: Dieses Werk ist unter einem Creative Commons Attribution-NonCommercial-ShareAlike 3.0 Germany Lizenzvertrag lizenziert. Um die Lizenz anzusehen, gehen Sie bitte zu http://creativecommons.org/licenses/by-nc-sa/3.0/de/ oder schicken Sie einen Brief an Creative Commons, 171 Second Street, Suite 300, San Francisco, California 94105, USA.
\section{Anreizsystem (Attribute)}
\subsubsection{Zweck}
Diese Regelerweiterung soll die Spieler für ein vom SL oder vom Spiel bestimmtes Verhalten belohnen. Hierbei kann es sich z.B. um emotionales, heroisches oder einfach nur besonders kreatives Rollenspiel drehen.

\subsubsection{Regelanpassungen:}
\subsubsection{Allgemeine Regeln}
Alle "`7"' gelten als farblos, d.h. sie können nicht der Attributsfarbe angehören.
\subsubsection{Charaktererschaffung}
Das zu belohnende Verhalten muss im Vorfeld definiert und den Spielern bekannt gegeben werden. Im einfachsten Fall reicht die Bekanntmachung aus, dass beispielsweise jeder selbstlose Einsatz für die Hilflosen und Schwachen von den Göttern gesegnet wird.Um komplexere Effekte zu erzielen, bietet sich jedoch die Bindung an die Werte des Charakters an. Jedem Attribut wird (von den Regeln der einBildung, dem SL oder dem Spieler nach Vorgaben des SL) eine Motivation oder ein Ziel mitgegeben, z.B.:
\begin{itemize}
\item Attribut: Stärke\\
Motivation: Für den König\\
Bedeutung: "`Für meinen König bin ich doppelt so stark"'\\
\\
      oder
\item Attribut: Charisma\\
Motivation: Minne\\
Bedeutung: "`Erst im Werben um eine holde Maid kommt mein Charme auf Touren"'
\end{itemize}
\subsubsection{Spielvorbereitung}
Die Spieler entfernen sämtliche Asse, Könige und "`7"' aus ihrem Deck und legen diese beiseite. Faire SLs tun das gleiche :-)
\subsubsection{Im Spiel}
Handelt ein Charakter gegen eine Motivation, muss er eine (in besonders schweren Fällen auch zwei) Siebener auf seinen Ablagestapel legen (wodurch sie nach dem nächsten Mischen in den Nachziehstapel gelangen). Wenn ein Charakter sich einer Motivation entsprechend verhält, darf der Spieler den König der entsprechenden Farbe auf seinen Ablagestapel legen. Hat er sich den entsprechenden König bereits verdient, darf er stattdessen das Ass nehmen. Ist auch das nicht mehr verfügbar, darf er irgendeinen König oder wenn er auch die bereits alle hat ein beliebiges Ass wählen. Folgte der SC seiner Motivation sogar unter besonderer Opferbereitschaft oder großem Risiko, darf er Ass und König auf den Ablagestapel legen. Alternativ darf er eine "`7"' zurück zu den ausgesonderten Karten legen. Asse und Könige, die in einem Konflikt ausgespielt wurden, kehren zurück zu den aussortierten Karten. Sind sie aber bis zum Ende des Konflikts auf der Hand verblieben, werden sie auf den Ablagestapel gelegt und erneut in den nächsten Nachziehstapel eingemischt.



