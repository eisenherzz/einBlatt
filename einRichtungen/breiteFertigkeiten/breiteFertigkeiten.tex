%einBlatt von Tempestrider & eisenherzz: Dieses Werk ist unter einem Creative Commons Attribution-NonCommercial-ShareAlike 3.0 Germany Lizenzvertrag lizenziert. Um die Lizenz anzusehen, gehen Sie bitte zu http://creativecommons.org/licenses/by-nc-sa/3.0/de/ oder schicken Sie einen Brief an Creative Commons, 171 Second Street, Suite 300, San Francisco, California 94105, USA.
\section{Breite Fertigkeiten}
Mit dieser einRichtung können SCs grundlegende Fertigkeiten in weiten Bereichen erhalten.
\subsection{Problem:}
Nur in des seltensten Fällen wird ein SL eine Fertigkeit wie beispielsweise "`Bewaffneter Nahkampf"' erlauben -- sie ist einfach zu mächtig. Andererseits kann kein Krieger, selbst mit einem Attributswert von 10, alle Waffengattungen und Kampfstile als Fertigkeiten auflisten.

\subsection{In der Charaktererschaffung:}
Um breite Grundfähigkeiten abzubilden, können diese ihrer Breite entsprechend mit einem Malus versehen werden, z.B.:
\begin{itemize}
\item Klingenwaffen: -2
\item Nahkampfwaffen: -3
\item Nahkampf: -4
\item Kämpfen: -5
\end{itemize}
Diese Fertigkeiten können mehrmals genommen werden, wodurch der Malus jeweils um 1 sinkt. Die Fertigkeit "`Kämpfen: -3"' würde also drei Fertigkeiten entsprechen (einmal um sie auf -5 zu nehmen und zwei weitere um sie auf -3 anzuheben). Unter -2 kann der Malus nicht gesenkt werden.

\subsection{In Proben:}
Verwendet ein SC in einem Konflikt eine breite Fertigkeit, zieht er den Malus von der Anzahl der Karten, die er zieht, ab.


