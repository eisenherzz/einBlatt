%einBlatt von Tempestrider & eisenherzz: Dieses Werk ist unter einem Creative Commons Attribution-NonCommercial-ShareAlike 3.0 Germany Lizenzvertrag lizenziert. Um die Lizenz anzusehen, gehen Sie bitte zu http://creativecommons.org/licenses/by-nc-sa/3.0/de/ oder schicken Sie einen Brief an Creative Commons, 171 Second Street, Suite 300, San Francisco, California 94105, USA.
\section{Zauberwörter Magiesystem}
\subsubsection{Voraussetzungen}
\begin{enumerate}
\item Es wird ein Attribut benötigt, von dem die magischen Fähigkeiten des SCs abhängig gemacht werden (z.B.: Weisheit, Mana, \dots). Da die Zauber über Fertigkeiten an diesem Attribut gebildet werden, müssen Magier sich zwischen ihren Zaubern und den weniger magischen Fertigkeiten entscheiden.
\item Falls gewünscht muss festgelegt werden, welchen Preis das Magietalent hat.
\end{enumerate}
\subsubsection{Charaktererschaffung}

Charaktere mit magischen Fähigkeiten verfügen unter dem Magieattribut statt Fertigkeiten über "`Zauberwörter"', die folgenden Kategoreien angehören:
\begin{itemize}
\item Prinzipien (Objekte, auf die ein Zauber wirken kann; Substantive)
\item Prozesse (Wirkungen von Zaubern; Verben)
\item Qualitäten (Modifikatoren für die Magie selbst; Adjektive)
\end{itemize}
All diese Zauberwörter werden weiter unten beschrieben.

Die Anzahl der Prinzipien und der Prozesse, über die ein SC verfügt, darf maximal um 2 auseinander liegen. Die der Qualitäten darf er frei bestimmen. Vom SL als "`mächtig"' eingestufte Zauberwörter müssen mit einem, "`besonders mächtige"' mit zwei Sternen markiert werden.

\subsection{Funktionsweise der Magie}

 
\subsubsection{Allgemeine Vorbemerkung}

Dieses Magiesystem ist sehr flexibel und interpretierbar. Einerseits bedeutet das, dass der Spieler einen großen kreativen Freiraum genießt. Andererseits bedeutet es aber auch, dass die Magie ihm nur selten völlig gehorchen wird -- sie ist eine eigene, freie Kraft, die genutzt, aber nicht berechnet oder gar beherrscht werden kann. In allem hat der SL das letzte Wort -- und ganz explizit das Recht, zu tun, was er will! Als Ratschlag sollte hier das "`Ja, aber \dots"' - Prinzip gelten: Der Zauber, den der Spieler angestrebt hat, wird gewirkt, aber je nach Ausgang der Zauberprobe passieren noch weitere unangenehmen Dinge oder gehen Teile schief.

\subsubsection{Prinzipien und Prozesse}

Im einfachsten Fall besteht der Effekt eines Zaubers immer aus einem Prozess (einer Handlung, ausgedrückt als Verb) und einem betroffenen Prinzip (ein Gegenstand im aller weitesten Sinne, ausgedrückt als Substantiv). So könnte man beispielsweise der allseits beliebte Flammenstrahl als "`Feuer bewegen"' beschrieben werden (mit dem Prinzip "`Feuer"' und dem Prozess "`bewegen"', ein Heilzauber als "`Fleisch stärken"' und ein Verhörzauber als "`Wahrheit erkennen"' -- oder, in einer weniger humanen Variante "`Lüge verbrennen"'. Die Einsatzmöglichkeiten von "`Dämonen rufen"' sollten selbsterklärend sein.
Unter Umständen kann ein Zauber auch zwei Prinzipien beinhalten, wobei das zweite als Ziel fungiert (es wird also dennoch genau ein Prozess benötigt). So könnte beispielsweise obiger Flammenstrahl um das Prinzip "`Feind"' zu "`Feuer auf Feind bewegen"' erweitert werden.

\subsubsection{Qualitäten}

Außerdem gibt es noch eine dritte Art von Zauberwörtern: Qualitäten, die das "`Wie"' eines Zaubers beschreiben. Mit Zustimmung des SL (die völlig willkürlich erteilt und widerrufen werden kann), dürfen Qualitäten auf Prinzipien oder Prozesse angewendet werden. So kann die Qualität "`vergangen"' Magie auf zurückliegende Ereignisse wirken lassen, "`geheim"' kann einen Zauber wie einen Zufall erscheinen lassen und nicht umsonst ist "`klebrig"' das erste Zauberwort aller kleinen Kobolde.
Ganz allgemein gilt, dass jedes Zauberwort vom SL abgesegnet werden muss.

\subsection{Wirkung und Grenzen von Magie}
\subsubsection{Reichweite}

Die meisten Zauberwörter wirken auf Kontakt. Dort, wo es offensichtlich sinnvoll ist gilt die natürliche Sichtweite als Reichweite. Auf der Zeitachse wirkt Magie jetzt und auf die Gegenwart.

\subsubsection{Wirkungsdauer}

Die Wirkungsdauer bestimmt in aller Willkür (aber bevor der Zauber gesprochen wird) der SL -- sie mag reichen von "`Zauberattribut in Sekunden"' über "`bis es jemand ändert"' bis hin zu "`für immer"'. Nur wenige Zauber haben jedoch eine große Lebensdauer.

\subsubsection{Grenzen verschieben}

Qualitäten wie "`künftig"', "`weit"' oder "`langlebig"' können Zauber jenseits dieser Grenzen ermöglichen. Hiervon wird jedoch im allgemeinen abgeraten -- zu groß sind die Risiken\dots
Ansonsten kann der SL Grenzverschiebungen anbieten und im Gegenzug die Zauberprobe schwerer machen -- hierzu ist er jedoch nicht verpflichtet.


\subsection{Die Gestaltung von Prozessen und Prinzipien}

\subsubsection{Breite contra enge Zauberwörter}

Die "`Breite eines Zauberworts"' bezeichnet die Größe seines Anwendungsbereichs. "`Wesen"' hat eine größere Breite als "`Mensch"', das wiederum breiter ist als "`Freund"' oder "`Feind"'. Die Breite hat folgende Bedeutung:
\begin{enumerate}
\item Ein breites Zauberwort wird der SC natürlich häufiger nutzen können
\item Andererseits kann er ein enges erheblich besser kontrollieren -- denn der Zauber ist stets nur genau so eng definiert, wie es seine Zauberworte tun. Eine Flammenlanze ("`Feuer bewegen"'), die um das Ziel "`Wesen"' erweitert wird, wird fast nie den nahenden schwarzen Ritter treffen -- zu viele Fliegen, Käfer und andere "`Wesen"' bieten sich als (aus Sicht der Magie) ebenbürtige Alternativen. Das Ziel "`Feind"' hingegen träfe wohl nur auf den schwarzen Ritter zu.
\item Den Nachteil breiter Worte können Qualitäten häufig lindern, wodurch aber der Zauber selbst natürlich wieder schwerer wird.
\end{enumerate}
 
\subsubsection{Keine "`flexiblen"' Zauberwörter}

Jeder, der auch nur einen Funken Powergamer in sich trägt, hat sich sicherlich bereits überlegt, welche Qualitäten aus breiten Prinzipien enge machen könnten. "`Bestimmt"' beispielsweise könnte doch aus eine Feuerlanze gegen Menschen eine Feuerlanze gegen "`bestimmte"' Menschen machen, oder?
Prinzipiell ja -- aber solange nicht ein Zauberwort diese Bestimmung vornimmt, tut es der SL allein, und wenn ein Zauberwort sagt, welche bestimmten Menschen, dann braucht man "`bestimmt"' nicht mehr. Es ist bedeutungslos und macht den Zauber nur schwerer.
Ähnlich sinnlos ist die Suche nach doppeldeutigen Zauberwörtern -- sie gelten immer nur in der ersten Bedeutung, in der sie verwendet wurden.

\subsubsection{Vorsilben vermeiden}

Besonders für Prozesse gilt, dass Vorsilben wenn irgend möglich zu vermeiden sind. So ist beispielsweise "`brennen"' ein sehr brauchbarer Prozess, "`verbrennen"' jedoch sollte der SL nicht zulassen. Der Zauber kann sein Ziel sicherlich in Flammen setzen -- ob es aber ganz und gar "`verbrennt"', hängt auch von vielen anderen Faktoren ab.
Wörter, bei denen der Sinn (und nicht nur das "`Boah!"' des Effekts) von der Vorsilbe abhängt, sind hiervon natürlich ausgenommen -- der Prozess "`wandeln"' kann das "`verwandeln"' nun mal nicht ersetzen\dots

\subsubsection{Schwarze Magie}

Seit Jahrhunderten schon werden die mächtigsten Zauberwörter unter Verschluss gehalten und ihr Gebrauch (den jeder Magier in weitem Umkreis sofort spürt) geächtet. Auf dem Codex der Unwörter stehen unter anderem:
\begin{itemize}
\item töten / sterben / leben / Tod / Leben / etc.
\item immer
\item ewig
\item überall
\item jeder / alle / viele / etc.
\item Zeit
\item Raum
\item Mensch / Person /
\item alles weitere, was der SL auf die Liste setzen möchte
\end{itemize}
Ebenso gilt gilt jeder zauber, der ein Lebewesen direkt tötet (und nicht nur eine tödliche Verletzung hervorruft) als schwarze Magie.

\subsection{Zauber wirken}

\subsubsection{Die Magieprobe}

Will ein SC einen Zauber wirken, so benennt er alle beteiligten Zauberwörter und einigt sich mit dem SL über den Effekt, den er damit erreichen möchte. Will der SC ein Prinzip einsetzen, über dass er nicht verfügt, so gilt dies als automatisch "`mächtig"' und natürlich als Einsatz einer nicht vorhandenen Fertigkeit (es darf also nur der halbe Attributswert verwendet werden). Pro Zauber darf maximal ein unbekanntes Wort verwendet werden. Wörter, die der SL als "`mächtig"' oder als "`besonders mächtig"' einstuft, dürfen nicht eingesetzt werden, ohne dass der SC sie beherrscht. Dann gibt der SL die Schwierigkeit bekannt. Diese ergibt sich wie folgt:
\begin{itemize}
\item jedes Zauberwort: +1
\item jedes als "`mächtig"' oder "`besonders mächtig"' eingestufte Zauberwort: zusätzlich +1
\item Der eigentliche Effekt des Zaubers: +1 (eine Feder schweben lassen) -- +5 (eine Stadtmauer sprengen).
\item Der SL kann dem Spieler anbieten, Wirkungsdauer, Reichweite etc. für eine höhere Schwierigkeit zu verändern
\end{itemize}

Die Attributsfarben, die der SL gegen den Zauber des SC wertet, ergeben sich wie folgt:
\begin{itemize}
\item Karo ist immer Attributsfarbe gegen den Magier
\item Herz wenn er schwarze Magie einsetzt
\item Kreuz wenn er mindestens zwei "`mächtige"' oder ein "`besonders mächtiges"' Zauberwort verwendet
\item Pik, wenn er mindestens zwei "`besonders mächtige"' Zauberworte verwendet
\end{itemize}
Erreicht der SC lediglich einen knappen Erfolg, wird wenigstens eines der Zauberworte eine "`unvorhergesehene"' Wirkung entfalten, der Zauber wird aber trotzdem keinen Schaden anrichten.\\
\\
Für einem klaren Erfolg bekommt der SC recht genau den angestrebten Effekt.\\
\\
Ein vollkommener Erfolg dehnt die Grenzen des Zaubers -- er trifft mehrere Ziele auf einmal, wirkt länger, stärker, etc., aber alles im Sinne des Magiers. 

In allen drei Fällen ist der Spieler selbst angehalten, Vorschläge für die genaue Wirkung zu machen.

\subsubsection{Die Widerstandsprobe}

Ist der Zauber gegen jemanden oder etwas gerichtet, legt der SL vor Zauberprobe fest, mit welchem Attribut und ggf. welcher Fertigkeit das Opfer versuchen darf, dem Zauber zu widerstehen. Um dies zu tun, tritt es als dritte Partei in die Zauberprobe ein.
Es sollten nur Fertigkeiten angewendet werden dürfen, die sehr genau passen -- oder eine spezifische Magieresistenz-Fertigkeit.

\subsubsection{Die Requisitenprobe}

Häufig wird ein Magier einen Teil eines Zaubers nicht mit seinen Zauberworten abdecken wollen, sondern weltlichere Lösung suchen. Hierfür muss eine Probe auf eine entsprechende nicht-magische Fertigkeit zusätzlich zu der Zauberprobe abgelegt werden. Viele Feuermagier erzeugen und bewegen ihre Feuerbälle magisch, zielen jedoch wie es jeder Bogenschütze tut.

\subsection{Beispiele}
\subsubsection{Voraussetzungen}

 
\subsubsection{Das Magie-Attribut}

Bei Verwendung der einRichtung "`einfache Fantasy Attribute"'(siehe Kapitel \ref {sect:einfacheFantasyAttribute}, Seite \pageref {sect:einfacheFantasyAttribute}) (Stärke, Geschick, Weisheit, Charisma) bietet sich natürlich Weisheit als Magie-Attribut an. Dies hat jedoch auch Nachteile: Unter dem Magieattribut werden Magier für gewöhnlich nur wenig anderes als ihre Magie-Fertigkeiten haben. Weisheit zum Magie-Attribut zu machen sorgt als für Magier, die außer vom Zaubern von nichts eine Ahnung haben -- Gandalf wäre in diesem System nicht abbildbar.
Eine Alternative besteht in Charisma: Dies lässt Magiern einerseits die Möglichkeit, fremde Sprachen zu lernen und Wissen über okkulte Traditionen zu sammeln, während es andererseits erklärt, warum fast alle Magier auf ihre Mitmenschen wie komplette Nerds wirken.

Welche Lösung hier die bessere ist (oder ob es in Deinem Setting besser passt, Magie an Stärke zu binden), muss allerdings immer von Fall zu Fall entschieden werden.
Der Preis des Magietalents

Es besteht eine Reihe verschiedener Möglichkeiten. Hier einige Beispiele:
\begin{itemize}
\item Keiner: Jeder Charakter darf sich magische Fähigkeiten nehmen
\item Attributsanforderungen: ein SC muss gewisse Mindestwerte in bestimmten Attributen erfüllen, um magische Fähigkeiten besitzen zu können
\item Attributsbeschränkungen: magiebegabte SCs dürfen in den körperlichen Attributen gewisse Maximalwerte nicht überschreiten
\item Magische Nachteile: für jede magische Fertigkeit muss der SCs einen magischen Nachteil erhalten, beispielsweise:
\begin{itemize} 
\item Der SC wird von bösen Feen als Spielzeug begehrt
\item Er ist ständig von Salpeterduft umgeben, außer eine hässliche Frau berührt ihn gerade (Charme - 1)
\item Er hat eintreibungswürdige Schulden bei der Magierakademie, die er Hals über Kopf verlassen hat, weil er nicht glaubte, die Prüfung bestehen zu können
\end{itemize}
\end{itemize}

\subsubsection{Funktionsweise der Magie}

\subsubsection{Prinzipien}
\begin{itemize} 
\item Feuer*, Wasser*, Erde*, Luft*
\item Lüge
\item Licht
\item Gedanke**
\item Kraft*
\item Weg*
\item Liebe**
\item Gier
\item Blut**
\item Schmerz
\end{itemize}

\subsubsection{Prozesse}
\begin{itemize} 
\item erkennen
\item bewegen
\item heilen
\item zerstören**
\item anhalten*
\item verschwinden*
\item verwandeln**
\item betören
\end{itemize}

\subsubsection{Qualitäten}
\begin{itemize} 
\item früher
\item schön
\item morgen
\item schnell
\item leise
\end{itemize}

\subsubsection{Wirkung und Grenzen von Magie}
\subsubsection{Reichweite}

Der Zauber "`Wunde heilen"' hat eine üblicherweise Reichweite von "`Berührung"'.
Der Zauber "`Feuer erschaffen bewegen"' (einen Feuerball erzeugen und schleudern) hat normalerweise eine Reichweite "`Sichtweite"'.
Der Zauber "`Lüge erkennen"' kann nur einen Betrug aufdecken, der im Moment des Zaubers oder wenige Minuten danach erfolgt. Um eine zurückliegende Aussage zu überprüfen, müsste der Magier eine Qualität wie "`vergangen"' einflechten -- und wenn er sich nicht am Ort der Aussage befindet oder den  Verdächtigen direkt mit dem Zauber belegen kann, benötigt er außerdem eine Qualität wie "`entfernt"'.

\subsubsection{Wirkungsdauer}

Hier sollte wirklich jeder SL seinen Weg finden.

\subsubsection{Grenzen verschieben}
Siehe Reichweite.

\subsubsection{Schwarze Magie}

Ein junger Adept hat im Labor seines Mentors ein paar Wunderlampen zu viel geputzt und soll nun zur Strafe das Abendessen für den ganzen Konvent richten. Doch schon beim Schlachten des Schweins verlässt ihn der Mut und er verlegt sich wieder auf eine Magische Lösung. Er berührt also das Schwein, wirkt seinen Zauber "`Blut** verschwinden"' und wird von einem wütenden, talarbekleideten Mob geächtet und davon gejagt, da sein Zauber verboten ist (da sein Zauber nur zu töten vermag). Zur gleichen Zeit entledigt sich sein leicht aufbrausender Mentor mittels eines Feuerballs einer Schnake. Auch er wird vom Hof gejagt, aber nur wegen der angesengten Tagebücher des Erzmagus -- ein Feuerball tötet nicht, er verwundet nur. Dass selbst ein hundert mal größeres Insekt den Feuerball nicht hätte überleben können spielt hierbei keine Rolle -- es geht nur um die prinzipielle Wirkungsweise des Zaubers.


\subsubsection{Zauber wirken}

\subsubsection{Die Magieprobe}

Ein SC will mit den Worten "`Feind"', "`Atem*"' und "`anhalten"' den Oberbösewicht aus der Welt schaffen. Leider isst der ein ganzes Stück weit weg und der SC muss das Wort "`entfernt"' improvisieren. So ergibt sich als Schwierigkeit:
\begin{itemize}
\item 4 Zauberwörter: +4
\item "`Atem*"' und das improvisierte "`entfernt"' = 2 "`mächtige"' Zauberwörter: +2
\item Der Effekt (nach Bewertung des SL): +4
\item Der SL zieht also 10 Karten -- nicht gerade ermutigend\dots
\end{itemize}
Nun zu den Attributsfarben:
\begin{itemize}
\item Karo ist immer Attributsfarbe gegen den Magier, also auch hier.
\item da er schwarze Magie einsetzt, gilt auch Herz als Attributsfarbe für den SL
\item Wegen seiner zwei "`mächtigen"' Zauberworte gilt auch Kreuz
\item Aber immerhin bleibt ihm Pik erspart, da er nicht mal ein "`besonders mächtiges"' Zauberwort verwendet.
\end{itemize}
Ich würde diesen Zauber nicht versuchen\dots

\subsubsection{Die Widerstandsprobe}

Da unser Test-Magier aber nicht so klug isst wie ich, legt der SL nun fest, dass sein Gegner mit seiner Gesundheit dagegen halten kann, also mit dem Attribut "`Stärke"'. Eine geeignete Fertigkeit wie "`Luft anhalten"' oder auch nur "`Tauchen"' hat das Opfer nicht, also kann es nur mit den halben Attributswert einsetzen.
Trotzdem verbessert das die ohnehin schlechten Aussichten unseres übernatürlichen Optimisten nicht gerade\dots

\subsubsection{Die Requisitenprobe}

Nachdem dieser kleine Mordanschlag gescheitert und ein neuer Magier-Charakter erschaffen ist, arbeitet sich die Gruppe auf traditionelle Weise zum Endgegner vor. Dort versucht sich der Neuling mit einem klassischen Feuerball:
"`Feuer"', "`erschaffen"', "`bewegen"'. Die Probe gelingt ihm vollkommen -- das wäre das Ende für den Bösewicht. Doch gehörte das Zielen nicht zum Zauber, also bestimmt der SL, dass eine erfolgreiche Wurfwaffenprobe gegen die "`Ausweichen"'-Fertigkeit des Erzfeindes über den Erfolg entscheidet.

