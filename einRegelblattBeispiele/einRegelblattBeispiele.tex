%einBlatt von Tempestrider & eisenherzz: Dieses Werk ist unter einem Creative Commons Attribution-NonCommercial-ShareAlike 3.0 Germany Lizenzvertrag lizenziert. Um die Lizenz anzusehen, gehen Sie bitte zu http://creativecommons.org/licenses/by-nc-sa/3.0/de/ oder schicken Sie einen Brief an Creative Commons, 171 Second Street, Suite 300, San Francisco, California 94105, USA.
\section {Erläuterungen und Beispiele zu einRegelBlatt}
\subsection {Charaktererschaffung}
\subsubsection {Attribute}
In diesem Bereich bietet einBlatt viel Raum für Erweiterungen (einRichtungen und einBildungen). Es werden bewusst weder die Namen der Attribute oder Fertigkeiten, noch ein Mechanismus zur Punkteverteilung vorgegeben. Nehmen wir beispielsweise an, es wird eine einRichtung mit Standard-Fantasy Attributen verwendet, dann könnte das wie folgt aussehen:

\begin{table}[H]
\caption{Beispiel Attribute}
\label{tab:beispielattribute}
\begin{tabular}{|l|l|}
\hline
\kreuz & Stärke\\
\pik & Gewandheit\\
\karo & Weisheit\\
\herz & Ausstrahlung\\
\hline
\end{tabular}
\end{table}

Für die Charaktererschaffung könnte man nun sagen, dass einfach 23 Punkte pro SC zu verteilen sind. Will man Spezialisierung bei den Charakteren fördern, könnte man stattdessen ein System wie das folgende anwenden:
\\
Die Spieler erhalten 61 Punkte um Attribute zu kaufen. Vor der Punkteverteilung benennt jeder ein starkes Attribut für seinen SC. Punkte in diesem Attribut kosten jeweils 2 Einkaufspunkte, in allen anderen jedoch 3.
\\
\\
Nun bauen wir mit beiden Systemen einen typischen Krieger. Mit den 23 Punkten könnte das so aussehen:

\begin{table}[H]
\caption{Beispiel Punkteverteilung}
\label{tab:beispielpunkteverteilung}
\begin{tabular}{|l|l|l|}
\hline
\kreuz & Stärke & 8\\
\pik & Gewandheit & 7\\
\karo & Weisheit & 3\\
\herz & Ausstrahlung & 5\\
\hline
\end{tabular}
\end{table}

Macht in der Summe (8+7+3+5) 23. 
\\
Und nun das ganze mit dem etwas komplizierteren System (natürlich ist Stärke das starke Attribut):

\begin{table}[H]
\caption{Beispiel Punkteverteilung Spezialisierung}
\label{tab:beispielpunkteverteilungspezialisierung}
\begin{tabular}{|l|l|l|}
\hline
\kreuz & Stärke & 10 (10 x 2=20 Einkaufspunkte)\\
\pik & Gewandheit & 8 (8 x 3=24 Einkaufspunkte)\\
\karo & Weisheit & 2 (2 x 3=6 Einkaufspunkte)\\
\herz & Ausstrahlung & 4 (4 x 3=12 Einkaufspunkte)\\
\hline
\end{tabular}
\end{table}

Macht in der Summe 62 -- und mit Stärke 10 bildet sich unser "`Spezialist"' ein, dass es auf den einen verschenkten Punkt nicht ankommt.
\\
\\
Es ist nicht nur möglich, den "`Normalbereich"' weiter hinter sich zu lassen als in diesem Beispiel -- es ist explizit erwünscht. Stellen wir uns ein Spiel vor, in dem die SCs Geister sind, die selbst kaum Einfluss auf die physische Welt nehmen können, die aber Macht über Menschen gewinnen können, indem sie bestimmte Gefühle in ihnen wecken. Hierzu haben sie die vier Emotionsattribute Angst (\pik), Zorn (\karo), Leid (\kreuz) und Freude (\herz), mit denen sie Opfer, in denen sie diese Gefühle geweckt haben, unter ihre Kontrolle bringen, sowie ein Attribut "`Projektion"', mit dem sie mit Menschen kommunizieren können (die Fertigkeiten bestimmen, auf welche Art und Weise). Um deutlich zu machen, wie schwierig die Einflussnahme aus dem Reich der Toten ist, kommt dieses fünfte Attribut ohne eine Attributsfarbe aus.

Einige andere Möglichkeiten:
\begin {itemize}
\item Attribute, denen mehrere Farben zugeordnet werden
\item Attribute, für die in Proben keine roten oder schwarzen Karten gespielt werden dürfen
\item Ein frei definierbares Attribut (beispielsweise für übernatürliche Fähigkeiten)
\item \dots
\end {itemize}

\subsubsection {Fertigkeiten}
In aller Regel sollte einBlatt ohne konkrete Listen von Fertigkeiten auskommen. Was als Fertigkeit zulässig ist und was nicht kann ein erfahrener SL meist besser entscheiden als ein Regelwerk. Außerdem ist das freie benennen von Fertigkeiten eine gute Möglichkeit, einem SC individuelle Färbung zu verleihen. Ob eine Fertigkeit als Verb, Substantiv oder Adjektiv, in einem oder mehreren Worten beschrieben wird, bleibt dem Spieler überlassen.
Kehren wir zurück zu unserem Krieger von vorhin (1. Beispiel) und geben wir ihm für seine Stärke von acht Punkten acht Fertigkeiten:

\begin{table}[H]
\caption{Beispiel stärkebasierte Fertigkeiten}
%\label{tab:beispielstärkebasiertefertigkeiten}
\begin{tabular}{|l|l|l|l|}
\hline
Äxte & Schwerter & Faustkampf & Ausdauer\\
\hline
Trinkfestigkeit & Gesundheit & Rammen & Muskulöse Erscheinung\\
\hline
\end{tabular}
\end{table}

Klar, kann man so machen. 0-8/15 eben. Es geht aber auch so :

\begin {enumerate}
\item Zögling Eirik Jarulfsons (der ein gefürchteter Axtkämpfer war)
\item Gyrischer Tempelgardist (eine Schwertkämpfer-Eliteeinheit)
\item Rauflustiger Trunkenbold
\item Sleipnirs Lungen
\item Trinkfester Raufbold
\item "`Ist nur ein Kratzer"'
\item Öffnet Türen mit der Schulter
\item Nackt am schönsten
\end {enumerate}

Manche dieser Definitionen sind dem Wortlaut nach enger als die klassischen, andere breiter. Das muss im Spiel nicht unbedingt so gehandhabt werden -- aber es ist wichtig, dass Spieler und SL sich über die Grenzen der Fertigkeit einig sind und dass die Fertigkeiten insgesamt einigermaßen ausgewogen bleiben. Es sollte nicht um möglichst mächtige Fertigkeiten gehen, sondern um solche, die ein klareres Bild vom SC zeichnen.

\subsubsection {Charaktererschaffung von A bis Z}
Hendrik will sich für eine Fantasy-Runde den Thorwinger Superhendrik erschaffen. Der SL hat hierfür eine Erweiterung ausgewählt, die auf folgenden Attribute basiert:

\begin{table}[H]
\caption{Superhendriks Attribute}
\label{tab:superhendriksattribute}
\begin{tabular}{|l|l|}
\hline
\karo & Kopf (Intelligenz, Wahrnehmung, Gedächtnis, \dots)\\
\herz & Herz (Menschenkenntnis, Charisma, Mut, \dots)\\
\kreuz & Körper (Stärke, Geschick, Ausdauer, \dots)\\
\pik & Ahnen (Verbindung zu den eigenen Vorfahren)\\
\hline
\end{tabular}
\end{table}

Das Setting sieht vor, dass jeder Charakter 21 Punkte auf die Attribute verteilen darf, also sieht Superhendrik nach zwei Minuten so aus:

\begin{table}[H]
\caption{Superhendriks Attribute (mit Werten)}
\label{tab:superhendriksattributemitwerten}
\begin{tabular}{|l|l|l|}
\hline
\karo & Kopf & 3\\
\herz & Herz & 5\\
\kreuz & Körper & 7\\
\pik & Ahnen & 5\\
\hline
\end{tabular}
\end{table}

Nun macht er sich an die Fertigkeiten. Zuerst Kopf (das geht am schnellsten, weil er da nur 3 Stück braucht). Dann Herz -- das sind schon 5. Der Körper bekommt zwar 7 Fertigkeiten, doch die werden Hendrik schnell von der Hand gehen. Die 5 Ahnen von Superhendrik sind folgende:
\begin{table}[H]
\caption{Superhendriks Fertigkeiten}
\label{tab:superhendriksfertigkeiten}
\begin{tabular}{|l|l|l|p{7cm}|}
\hline
\karo Kopf 3& \herz Herz 5& \kreuz Körper 7& \pik Ahnen 5\\
\hline
Orientierung & Einschüchtern & Axt & Inna (seine Großmutter, die berühmt wurde, weil sie mit bloßen Händen Wölfe zur Strecke brachte) \\
Tierkunde & Lügen ahnen & Schild & Ragnulf (ein Skalde, der vor 100 Jahren starb, dessen Lieder aber noch heute gesungen werden) \\
Orkische Sprache & Männer führen & Schwert & Ulf (sein Großonkel und der beste Seemann seiner Zeit) \\
& Mut & Speer & Ingmar (der einzige aus seiner Familie, der jemals Hetmann seiner eigenen Otta war) \\
& Mitreißend singen & Reiten & Hildrun (Eine Hexe, die vor 700 Jahren einen Halbgott hereinlegte, um mit ihm Superhendriks Familie zu zeugen) \\
& & Raufen &\\
& & Saufen &\\
\hline
\end{tabular}
\end{table}

Dann radiert Hendrik Superhendriks Namen aus und nennt ihn Kent -- das passt irgendwie besser in die Familie.

Damit nicht alle Probleme dieser Runde mit dem Schwert gelöst werden, bastelt sich Stefan einen alten Druiden namens Eirik, der Kent begleiten soll:
\begin{table}[H]
\caption{Eiriks Fertigkeiten}
\label{tab:eiriksfertigkeiten}
\begin{tabular}{|l|l|l|p{7cm}|}
\hline
\karo Kopf 9& \herz Herz 3& \kreuz Körper 6& \pik Ahnen 3\\
\hline
Fährten lesen & Einschüchtern & Schleichen & Stig (der erfolgreichste Trickbetrüger, der je in Thorwingen gelebt hat) \\
Pflanzenkunde & Lügen erkennen & Stabkampf & Ole (der insgeheim unter jedem Dach Thorwingens Kinder hatte) \\
Heilkunde & Lügen & Trinken & Lasse (Eiriks Großvater, der mit Karten- und Würfelspielen ein Vermögen gemacht hat) \\
Wahrsagerei &  & Reflexe &  \\
Eismagie &  & Rennen &  \\
Beherrschungsmagie & & Schwimmen  &\\
Schutzmagie & &  &\\
Heilungsmagie & & &\\
Axt & & &\\
\hline
\end{tabular}
\end{table}

"`Was soll die Axt denn an Deinem Kopf?"' fragt Hendrik. "`Erstens weiß ich viel über die und zweitens kann ich so immer noch ein bisschen kämpfen."' antwortet Stefan.

"`Du bist nicht mutig?"' 
"`Ich behaupte einfach, ich wäre es."'erwidert Stefan.


"`Schwimmen?"' Hendrik schaut besorgt auf sein Charakterblatt\dots
\\
\\
Zufrieden reicht Stefan Hendrik die Hand:
"`Freunde?"' 

"`Dein Feind will ich jedenfalls nicht sein\dots"'


\subsection {Zufall}
\subsubsection {Gültigkeit von Attributsfarben}

Wenn Karten miteinander verglichen werden sollen, stellt sich zunächst die Frage, ob ein Attribut und damit die zugehörige Farbe das Ergebnis beeinflussen könnte.

Beispiele:
\begin {itemize}
\item Der SL will auf einer Reise das Wetter vom Zufall abhängig machen. Er lässt einen Spieler eine Karte für gutes Wetter ziehen und zieht selbst eine für schlechtes. Sofern die SCs keine Halbgötter oder zumindest mächtige Schamanen sind, haben sie keinen Einfluss auf den Ausgang, also ignoriert der SL die Kartenfarben.
\item Ein SC will eine steile Felswand hinaufklettern. Der SL legt fest, dass er dazu eine Probe auf "`Gewandtheit + Klettern"' ablegen muss und setzt die Schwierigkeit der Felswand mit sieben an (zieht also sieben Karten). Der Spieler hat zwar zwei Karten weniger, aber da die Probe auf Gewandtheit geht, kann er die Attributsfarbe von Gewanddtheit für sich geltend machen, während die Felswand keine Attributsfarbe für sich anführen kann.
\item Der SL will ohne eine aufwändige Probe prüfen, welchem der SCs die Geheimtür auffällt, hinter der der restliche Plot lauert, die aber irgendwie keiner der Spieler wirklich sucht. Hierbei spielt natürlich die Wahrnehmung eine Rolle, also könnte die entsprechende Attributsfarbe stechen. Da das aber gleichermaßen für alle gelten würde, ignoriert der SL die Farben und die Regeln in diesem Fall am besten einfach.
\item Die SCs sind in einen schlimmen Sturm geraten und müssen sich mit Gewandtheitsproben in Sicherheit bringen. Um die Dimensionen des Sturmes deutlich zu machen legt der SL fest, dass die Spieler zwar ihre Gewandtheits-Attributsfarbe einsetzen dürfen, der Sturm seine aber auch -- alles außer \herz\dots
\end {itemize}

\subsubsection {Kartenvergleich}

Kleiner Tip: Um bei diesen Beispielen den Überblick zu behalten ist es ratsam, sie beim Lesen mit Skat-Karten "`mitzuspielen"'.

Sofern Attributsfarben anwendbar sind, entscheiden diese über den Ausgang des Vergleichs. Nur wenn mehrere oder gar keiner der Beteiligten ihre jeweiligen Attributsfarben gespielt haben, wird der Kartenwert verglichen.
\\
\\
Beispiel:

Um festzustellen, ob die SCs den in einem Gebüsch versteckten feindlichen Späher entdecken. lässt der SL alle Spieler Karten entsprechend ihres "`Wahrnehmung"'-Attributwertes ziehen, von denen sie dann eine spielen sollen. Die Attributsfarbe (in diesem bei allen \karo) gilt ebenso wie die des Gewandtheitsattributs des Spähers (\pik), für den der SL sechs Karten zieht.
\begin {itemize}
\item Spieler eins spielt ein \kreuz-Ass.
\item Spieler zwei spielt eine \herz-Dame.
\item Spieler drei spielt eine \karo-7 (*).
\item Spieler vier spielt eine \karo-10 (*).
\item Und der SL spielt eine \pik-10 (*).
\end {itemize}
(*) = Attributsfarbe

Damit haben drei Konfliktteilnehmer Attributsfarbe gespielt (Spieler drei, Spieler vier und der SL). Zwischen diesen entscheidet nun die Kartenhöhe. Die \karo-7 von Spieler drei ist die niedrigste der drei Karten. Die \karo-10 von Spieler vier jedoch ist gleichrangig mit der \pik-10 des Spähers, da beide ihre jeweilige Attributsfarbe gespielt haben. So wird der Feind zwar entdeckt, bemerkt dies aber im selben Augenblick und nimmt so einen kleinen Vorsprung mit auf die Flucht.


\subsection {Proben}
\subsubsection {Einfacher Fall: 1 gegen 1}

Beispiel 1 -- eine ganz normale Probe:

Billy will einen Baum hinauf klettern. Der SL bestimmt, dass der Konflikt zwischen Billy und dem Baum stattfindet. Für Billy legt er Gewandtheit:Klettern als Probenattribut und -Fertigkeit fest. Billy hat Gewandtheit 6 (Attributsfarbe \kreuz) und verfügt über die Fertigkeit Gewandtheit:Klettern, also zieht er sechs Karten. Als Schwierigkeit zieht der SL fünf Karten, legt aber keine Attributsfarbe fest.

Da Billy die benötigte Fertigkeit hat und sie dem Probenattribut zugehörig ist, darf er seinen vollen Attributswert ziehen -- also sechs Karten:
\begin{table}[H]
\caption{Beispiel 1 (Vorbereitung): Einfacher Fall: 1 gegen 1}
\label{tab:Beispiel1Vorber1gegen1}
\begin{tabular}{|l|l|}
\hline
Billy & SL\\
\hline
\herz: 8, König & \herz: 10 \\
\karo: 8, 9 & \karo: 7 \\
\pik: Bube, Dame & \pik: Dame\\
\kreuz: 10 & \kreuz: 8, Ass\\
\hline
\end{tabular}
\end{table}

\begin{table}[H]
\caption{Beispiel 1 (Rundenablauf): Einfacher Fall: 1 gegen 1}
\label{tab:Beispiel1Runden1gegen1}
\begin{tabular}{|p{5cm}|p{5cm}|p{5cm}|}
\hline
1. Runde & 2. Runde & 3. Runde\\
\hline
Billy: \herz-König & Billy: \pik-Dame & Billy: \kreuz-10\\
SL: \herz-10 & SL: \pik-Dame & SL: \kreuz-Ass\\
\hline
\multicolumn{3}{|c|}{Ergebnis}\\
\hline
Niemand hat Attributsfarbe gespielt. Billys König ist höher als die 10 des SL. Also gewinnt Billy. Billy hat 1 EP, der SL -1 & Niemand hat Attributsfarbe gespielt. Beide Karten sind gleich hoch. Also bekommt niemand Punkte. Billy hat 1 EP, der SL -1. & Billy hat seine Attributsfarbe gespielt, der SL nicht (weil er keine hat). Damit sind die Motive egal. Billy gewinnt diese Aktion. Billy hat 2 EP, der SL -2. Und laut Tabelle \ref {tab:konfliktergebnisse} (Seite \pageref {tab:konfliktergebnisse})  bedeutet das, dass Billy den Baum besteigt, ganz so, wie er sich das vorgestellt hat. \\
\hline
\end{tabular}
\end{table}

Beispiel 2:

Billys etwas bücherversessener Freund Saruman, der Jüngere, sieht Billy zu und beschließt, ihn zu übertreffen, Erst kürzlich hat er alles, was je über das Klettern geschrieben wurde, gelesen, und so schwer klang das gar nicht. "`Gibt's hier in der Nähe einen größeren Baum?"' fragt er den SL und der nickt. "`Da kletter ich rauf."'

Zu seiner Enttäuschung darf er aber nicht mit seiner Weisheit (Wert: 10, Attributsfarbe \karo) klettern, sondern muss ebenso wie Billy zuvor seine Gewandtheit (Wert: 4, Attributsfarbe \kreuz) proben. "`Ich kann aber auch Klettern!"' freut er sich. Aber da es nicht "`Gewandtheit:Klettern"' sondern "`Weisheit:Klettern"' ist, darf er nicht seine volle Gewandtheit einsetzen, sondern muss zwei davon abziehen (siehe Tabelle \ref {tab:probenattributprobenfertigkeitundhandkarten}, Seite \pageref {tab:probenattributprobenfertigkeitundhandkarten}). Also zieht er nur zwei Karten:

\begin{table}[H]
\caption{Beispiel 2 (Vorbereitung)}
\label{tab:Beispiel2Vorber}
\begin{tabular}{|l|l|}
\hline
Saruman & SL\\
\hline
& \herz: 7, 10, Dame \\
\karo: Bube & \karo: -- \\
& \pik: 7, 9, 10\\
\kreuz: 9 & \kreuz: Bube, Ass\\
\hline
\end{tabular}
\end{table}

"`Was machst denn Du da?"' fragt er den SL, als der seine Kartenhand aufnimmt. "`Ich ziehe für den Ent"' antwortet der SL, "`der hat übrigens \pik als Attributsfarbe für Gewandtheit, aber das merkst Du erst nachher\dots"'


\begin{table}[H]
\caption{Beispiel 2 (Rundenablauf)}
\label{tab:BeispielRunden1gegen1}
\begin{tabular}{|p{5cm}|p{5cm}|p{5cm}|}
\hline
1. Runde & 2. Runde & 3. Runde\\
\hline
Saruman: -- & Saruman: \kreuz-9 & Saruman: \karo-Bube\\
SL: \pik-9 & SL: \pik-7 & SL: \pik-10\\
\hline
\multicolumn{3}{|c|}{Ergebnis}\\
\hline
Der Ent spielt Attributsfarbe, Saruman spart seine Karten (= farblos 7), der Ent gewinnt. Ent: 1 EP, Saruman: -1 EP & Beide spielen ihre jeweilige Attributsfarbe. Sarumans 9 ist höher als die 7 des Ent. Saruman gewinnt. Ent: 0 EP, Saruman: 0 EP & Saruman spielt nicht Attributsfarbe, der Ent schon. Der Ent gewinnt. Ent: 1 EP, Saruman: -1 EP \\
\hline
\end{tabular}
\end{table}

Und mit diesem Endergebnis fällt Saruman von dem wütenden Baumhirten, kassiert bei der unsanften Landung eine 1-Punkt einSchränkung auf seine Gewandtheit und beschließt, dass er eines Tages entweder zaubern lernen muss um alle blöden Bäume mit Feuerlanzen einzuäschern, oder, wenn das nicht klappt, dass er viele doofe Orks anheuern muss, um dieses übellaunige Gestrüpp kamingerecht zerkleinern zu lassen\dots

\subsubsection {Etwas komplizierter: Konflikte mit mehreren Beteiligten}

Beispiel 3: SCs gegen NSCs

Die SCs Francis, Kyle und Mortimer liefern sich einen Seilzieh-Wettbewerb mit sechs Halbstarken aus dem Dorf, in dem sie auf die Ankunft des nächsten Plots warten. Der SL legt fest, dass die Stärke der SCs geprüft werden soll und fragt die Spieler, welche passenden Fertigkeiten sie anzubieten haben. Mortimer zuckt gleich mit den Schultern, während Kyle sein "`Standhaft wie ein Fels"' natürlich einsetzen darf. Francis schlägt sein Geschick:Athletik vor, was der SL aber zu breit ausgelegt findet und ablehnt.

"`Sonst nichts?"'

"`Vielleicht doch,"' meint Mortimers Spieler, "`Da die Abstimmung doch so wichtig ist beim Seilziehen, wie wäre es mit Ausstrahlung:Anführer?"'
Der SL zögert kurz und fragt dann, ob seine Freunde denn bereit wären, auf sein Kommando zu hören, was diese bejahen.

"`OK, aber es ist immer noch -2, weil es keine Stärke-Fertigkeit ist."'

Also zieht Francis (Stärke 7) 3 Karten (\(\frac{7}{2}\), abgerundet), Kyle (Stärke 6) zieht 6 und Mortimer (Stärke 7) zieht 5 \((7 - 2)\).
Der SL gesteht einem der Jungspunde 4 Karten zu, zwei 3 Karten, zwei 2 und dem "`Anführer"' 5.
So kommen folgende Kartenhände zustande:

\begin{table}[H]
\caption{Beispiel 3 (Vorbereitung)}
\label{tab:Beispiel3Vorber}
\begin{tabular}{|l|l|l|l|l|}
\hline
Charakter & \multicolumn{4}{|c|}{Kartenhand}\\
\hline
Francis &\herz: 9& & \pik: Bube, König&\\
Kyle & &\karo: 8, Ass  & \pik: 7, 10 &\kreuz: 8, Dame\\
Mortimer & \herz: König & & & \kreuz: 8, 9, 10, Dame\\
Junge 1 & \herz: Bube, Ass &\karo: 8 &\pik: 8 &\\
Junge 2 & & & \pik: 10 & \kreuz: 7, 10\\
Junge 3 & \herz: 8, 10, Dame & & &\\
Junge 4 & & & \pik: Bube & \kreuz: 9\\
Junge 5 & & \karo: 10 & \pik: 7 & \\
Anführer der Jungen & \herz: Ass &\karo: 9, König & &\kreuz: Ass\\
\hline
\end{tabular}
\end{table}
Die Attributsfarbe für alle ist \kreuz.

1. Runde:
\begin {itemize}
\item Francis: \pik-Bube
\item Kyle: \karo-Ass
\item Mortimer: \kreuz-8
\item Junge 1: \karo-8
\item Junge 2: \kreuz-10
\item Junge 3: \herz-8
\item Junge 4: \pik-Bube
\item Junge 5: Farblos-7 (passt)
\item Anführer der Jungen: \kreuz-Ass
\end {itemize}

Der Anführer gewinnt also die Initiative und wählt Kyle zu seinem Opfer. Sein \kreuz-Ass schlägt Kyles \karo-Ass, also gewinnt er einen EP, Kyle verliert einen und beide legen ihre eben verwendeten Karten ab. Als nächster ist der zweite Junge mit seiner \kreuz-10 an der Reihe -- und er greift Mortimer an. Wieder siegt der Angreifer und gewinnt einen EP, Mortimer verliert einen und beide legen ihre Karten auf den Ablagestapel. Somit hat Mortimer, der eigentlich die dritthöchste Karte gespielt hatte, nicht als nächster (oder genauer: überhaupt noch in dieser Runde) dran. Gleiches gilt für Kyle, dessen ebenfalls bereits abgelegtes \karo-Ass die vierthöchste Karte gewesen wäre. Die höchsten verbliebenen Karten sind die beiden \pik-Buben von Francis und dem vierten Jungen. Da Francis aber den höheren Attributswert hat, kommt er zuerst zum Zug.

"`Wen soll Francis angreifen?"' fragt der SL Mortimers Spieler.

"`Hallo, mein Charakter, hier!"' ruft Francis Spieler dazwischen, doch der SL entgegnet:

"`Ihr habt Mortimers Führung akzeptiert, sonst hätte er keine Karten für seine Anführer-Fertigkeit ziehen dürfen."' Er wendet sich wieder Mortimers Spieler zu: "`Also?"'

"`Den Anführer."'

"`Aber das ist doch dumm!"' ruft Francis Spieler, "`Der ist der letzte\dots"' 

"`Heißt das, Du greifst einen anderen an?"' fragt der SL.

"`Klar,"' antwortet Francis Spieler, "`ich nehme den ersten von den Jungs"'.

Francis erhält einen Punkt, der erste Junge verliert einen, die Karten landen auf den Ablagestapeln und der SL schaut sich Francis Karten an und lässt ihn die \kreuz-Dame abwerfen.

"`Und wenn nochmal jemand sein eigenes Ding dreht, machen wir das wieder."'

Außerdem greift der vierte Junge, der nun an der Reihe ist, Mortimer mit seinem \pik-Buben an.
Mortimer antwortet grinsend mit dem \herz-König von seiner Hand, gewinnt die Aktion und einen EP, Junge 4 verliert einen und König und Bube wandern auf die Ablagestapel.
Nun hat Junge 3 als Einziger noch eine Karte vor sich liegen -- eine \herz-8.

"`Damit würde ich mich einfach nur raushalten"' tönt Mortimers Spieler.

"`OK,"' sagt der SL,"`dann geht der auch gegen Dich."'

Mortimer antwortet mit seiner \kreuz-9, die zwar einen weiteren EP für ihn bedeutet, doch da ihm nun für die verbliebenen beiden Runden nur noch eine Karte bleibt (und auch die nur, wenn keiner seiner "`Freunde"' vorher wieder einen auf Selbstbestimmung macht), beschleicht ihn ein ungutes Gefühl\dots

Der Stand nach der ersten Runde:
\begin {itemize}
\item Francis: 1 EP (eine gewonnene Aktion)
\item Kyle: -1 EP (eine verlorene Aktion)
\item Mortimer: 1 EP (eine verlorene und zwei gewonnene Aktionen)
\item Junge 1: -1 EP
\item Junge 2: 1 EP
\item Junge 3: -1 EP
\item Junge 4: -1 EP
\item Junge 5: 0
\item Anführer der Jungen: 1 EP
\end {itemize}

2. Runde:
\begin {itemize}
\item Francis: \pik-König
\item Kyle: \kreuz-Dame
\item Mortimer: \kreuz-10
\item Junge 1: \pik-8
\item Junge 2: \pik-10
\item Junge 3: \herz-Dame
\item Junge 4: Farblos-7 (passt)
\item Junge 5: \pik-7
\item Anführer der Jungen: \karo-9
\end {itemize}

Nach dem Aufdecken klatschen die Spieler ab -- keiner der Jungs kann ihnen diesmal das Wasser reichen. Kyle eröffnet (auf Geheiß seines "`Anführers"' Mortimer) mit einem erfolgreichen Angriff auf den dritten Jungen, Mortimer selbst besiegt den gegnerischen Anführer und Francis die Nummer 2. Nach dieser Großoffensive haben nur noch die Jungs 1 und 5 ihre Karten vor sich liegen. Junge 1 greift Francis an, der sich aber mit seiner \herz-9 erfolgreich zur Wehr setzt, während Mortimer der \pik-7 des letzten Jungen gar keine Karte mehr entgegenzusetzen hat.

Der Stand nach der zweiten Runde:
\begin {itemize}
\item Francis: 3 EP
\item Kyle: 0 EP
\item Mortimer: 1 EP
\item Junge 1: -2 EP
\item Junge 2: 0 EP
\item Junge 3: -2 EP
\item Junge 4: -1 EP
\item Junge 5: 1 EP
\item Anführer der Jungen: 0 EP
\end {itemize}

3. Runde:
\begin {itemize}
\item Francis: Farblos-7 (passt)
\item Kyle: \kreuz-8
\item Mortimer: Farblos-7 (passt)
\item Junge 1: \herz-Ass
\item Junge 2: \kreuz-7
\item Junge 3: \herz-10
\item Junge 4: \kreuz-9
\item Junge 5: \karo-10
\item Anführer der Jungen: \herz-Ass
\end {itemize}

Junge 4 gewinnt die letzte Initiative und besiegt Kyle. Damit können die SCs selbst nicht mehr agieren. Von den fünf übrigen Jungs greifen 2 Mortimer und 3 Francis an, die sich beide nicht mehr wehren können.

So kommt folgendes Endergebenis zustande:
\begin {itemize}
\item Francis: 0 EP
\item Kyle: -1 EP
\item Mortimer: -1 EP
\item Junge 1: -1 EP
\item Junge 2: 1 EP
\item Junge 3: -1 EP
\item Junge 4: 0 EP
\item Junge 5: 2 EP
\item Anführer der Jungen: 1 EP
\end {itemize}

\subsubsection{Konfliktende}

In der Summe haben die Dorfkinder 2 EP mehr als die SCs, sie haben 3 Sieger (die SCs keinen), nur bei den Verlierern (jeweils 2) herrscht Gleichstand. Andererseits sind 2 EP Unterschied jetzt auch keine Demütigung -- eine deutliche Überlegenheit, mehr nicht.
Für die einzelnen SCs bedeutet dies, dass Kyle und Mortimer kleine Verletzungen eingesteckt haben und diese nun unter "`einSchränkungen"' auf ihren Charakterblättern eintragen müssen. Francis Spieler schreibt:

\begin{table}[H]
\begin{tabular}{|l|l|l|l|}
\hline
Datum & Attribut & Wert & Situation\\
8.10.2009 & Stärke & - 1 & beim Seilziehen mit der Dorfjugend Bizeps gezerrt\\
\hline
\end{tabular}
\end{table}

Auf Mortimers Charakterblatt stehen die gleichen Daten, nur als Beschreibung wählt er: "`Seit dem Seilziehen zwickt die alte Kriegsverletzung wieder"'.
\\
\\
Noch ein paar weitere Beispiele zur Interpretation der Ergebnisse von Proben:
\\
Vier der heldenhaften "`Ausgewürfelten 7"' wollen den abenteuerfreien Montag unklugerweise nicht mit Entspannung verbringen, sondern halten sich mit alltäglichen Übungen fit. Der generische Alrik klettert an seinem Lieblingsüberhang, doch wie es an Montagen so geschieht, legt er eine -1 Niederlage hin. Kein tiefer Sturz -- der SL versichert ihm, dass er den verlorenen Geschickpunkt wiederbekommt, sobald sein Knöchel verheilt ist. Alriks omnipotenter Kollege Loophole Maniac betätigt sich als Preisboxer, doch heute gerät er an den Falschen: er unterliegt mit -1, lässt ich dabei von seinem Gegner die Farbe seiner Augen und die Anzahl seiner Schneidezähne anpassen und verliert vorübergehend 1 Punkt auf seine Stärke.
Magus Maximus nutzt die Zeit lieber in seinem Labor, wo er sein Plot-Umgehungs-Artfekt endlich fertig stellen will. Leider legt er beim alles entscheidenden Magiewurf eine -3 hin. Ängstlich schaut M\&Ms Spieler den SL an.

"`Nicht schlimm, Du verlierst 4 Punkte von Deinem Magie-Attribut, kriegst aber jede Session einen Punkt zurück. Nur der vierte Punkt, den musst Du Dir irgendwie verdienen. Das Artefakt raucht ein wenig und blaue Amethyst hat einen Sprung."'

Das ist nicht so wild, den kann man schnell ersetzen -- Erleichtert notiert der Spieler seinen Schaden. Und der SL notiert, was dem Artefakt sonst noch passiert ist\dots


