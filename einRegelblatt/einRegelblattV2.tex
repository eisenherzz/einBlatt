%einBlatt von Tempestrider & eisenherzz: Dieses Werk ist unter einem Creative Commons Attribution-NonCommercial-ShareAlike 3.0 Germany Lizenzvertrag lizenziert. Um die Lizenz anzusehen, gehen Sie bitte zu http://creativecommons.org/licenses/by-nc-sa/3.0/de/ oder schicken Sie einen Brief an Creative Commons, 171 Second Street, Suite 300, San Francisco, California 94105, USA.
\part {einRegelblatt}
\chapter {einRegelblatt}
einRegelBlatt ist der Regelkern aller einBlatt-Systeme. Fragen, die einRegelBlatt nicht beantwortet, sind nicht als Lücken, sondern als Freiräume für Erweiterungen (einRichtungen, einBildungen und einGriffe) zu sehen. einRegelBlatt ist primär ausgerichtet auf Rollenspiele für einen Spielleiter ("`SL"') eine beliebige Anzahl von Spielern mit ihren Charakteren ("`SCs"'). Das Wort des SL ist Gesetz, egal was der gesunde Menschenverstand oder die Regeln sagen.
\section{Charaktererschaffung}
SCs in einBlatt haben in der Regel 4 Attribute, denen je
\begin{itemize}
\item eine Kartenfarbe des Skatblatts (\kreuz, \pik, \karo, \herz; von nun an "`Attributsfarbe"' genannt),
\item ein Wert von einmal 8 Punkten (Beruf), zweimal 6 Punkten (Hobby) und einmal 2 Punkten (Schwäche)
\end{itemize}
zugeordnet werden.

\subsection{Die Attribute}
\subsubsection{Beruf}
Ein Beruf ist ein Bündel von Tätigkeiten, das fachspezifische Kenntnisse und Fertigkeiten erfordert \footnote{\url{http://de.wikipedia.org/wiki/Beruf} Zugriff 20.11.2011}.
\subsubsection{Hobby}
Ein Hobby ist eine Lieblingsbeschäftigung. Ein Hobby ist somit im Gegensatz zu Arbeit eine Tätigkeit, der man sich nicht aus Notwendigkeit, sondern freiwillig und aus Interesse, Faszination oder sogar Leidenschaft unterzieht. Die Tätigkeit bringt Vergnügen, Spaß oder Lustgewinn mit sich. Dabei ist mit Arbeit nicht ausschließlich Erwerbsarbeit (Beruf) gemeint \footnote{\url{http://de.wikipedia.org/wiki/Hobby} Zugriff 20.11.2011}.
\subsubsection{Schwäche}
 
\section{Zufall}
einBlatt verwendet Skatkarten als Zufallselement. Jeder Spieler benutzt ein eigenes Skatblatt, einen eigenen Nachzieh- und Ablagestapel. Sobald ein Spieler seinen Nachziehstapel aufgebraucht hat, ersetzt er ihn durch den neu gemischten Ablagestapel. Bevor die Werte von Karten bestimmt und verglichen werden können, muss der SL festlegen, ob und ggf. welche Attribute das Ergebnis beeinflusssen (wie es z. B. in Proben immer der Fall ist). Karten in den jeweiligen Attributsfarben gelten dann automatisch als höher als alle anderen. Spielen die Attribute keine Rolle oder bringt der Farbvergleich kein Ergebnis, entscheidet das höhere Motiv (Reihenfolge: A-K-D-B-10-9-8-7).
\section {Proben}
\subsection {Allgemeiner Ablauf}
In einBlatt werden Proben als Wettkampf zwischen zwei oder mehr Konfliktteilnehmern (im Folgenden "`KT"' genannt; können natürliche Hindernisse sein) behandelt. Dieser besteht aus drei Runden, in denen alle Teilnehmer versuchen, Erfolgspunkte (EP) für gelungene Aktionen gegen ihre Kontrahenten zu sammeln. Aus der Summe ihrer EP am Ende des Konflikts ergibt sich das Ausmaß des (Miss-)Erfolgs -- 0 EP bedeuten ein unentschieden, eine negative Zahl ein Scheitern und eine positive einen Erfolg. Solange kein vollkommenes Ergebnis (+/- 3 EP) erzielt wurde, können die Beteiligten den Konflikt jedoch mit weiteren Proben fortsetzen.
\\
\\
Und jetzt noch mal im Detail:
\subsection {Konfliktvorbereitung: Handkarten ziehen}
Zu Beginn jeder Probe legt der SL zwei Dinge fest:
\begin{itemize}
\item für alle Nichstspieler-KT: eine Kartenzahl zwischen 1 (harmlos) und 32 (fast unbesiegbar)
\item für jeden beteiligten SC: das auf die Probe zu stellende Probenattribut
\end{itemize}
Aus diesem ergibt sich, wie viele Karten der Spieler auf die Hand nehmen darf (siehe Tabelle \ref {tab:probenattributundhandkarten}, Seite \pageref {tab:probenattributundhandkarten}). Im weiteren Verlauf des Konflikts werden keine weiteren Karten mehr gezogen.

\begin{table}[H]
\caption{Probenattribut und Handkarten}
\label{tab:probenattributundhandkarten}
\begin{tabular}{|l|l|}
\hline
Der SC nutzt das Probenattribut\dots & Der Spieler zieht\dots \\
\hline
Beruf & 8 \\
Hobby & 6 \\
gar nicht & 4 \\
Schwäche & 2 \\
\hline
\end{tabular}
\end{table}

\subsection {Konfliktabwicklung}
Zu Beginn jeder Runde spielen alle Konfliktteilnehmer gleichzeitig eine Karte ("`Rundenkarte"') aus ihrer Hand. Beginnend mit der höchsten führen die Spieler nun ihre Aktionen durch. Bei Gleichstand kommt der Teilnehmer mit dem höheren Attributswert zuerst an die Reihe. Herrscht auch hier Gleichstand, agieren beide gleichzeitig. Wenn ein KT an der Reihe ist, kann er:
\begin{itemize}
\item abwarten (\textasciitilde nichts tun; nächster Spieler ist dran)
\item agieren (SL entscheidet, wer wen angreifen darf)
\end{itemize}
Entscheidet er sich für die "`Aktion"', benennt er einen Gegner, der "`reagieren"' muss. Mit diesem vergleicht er die Rundenkarten. Hat der Angegriffene seine Rundenkarte bereits abgeworfen (weil er in dieser Runde schon einmal "`reagiert"' oder selbst "`agiert"' hat), spielt er eine neue aus seiner Hand. Fehlt die Rundenkarte aber, weil der Angegriffene KT "`gepasst"' (siehe Seite \pageref {subsect:passen}) hat, darf er keine neue Karte spielen. Der Teilnehmer mit der höheren Karte gewinnt die Aktion und einen EP, der Unterlegene verliert einen. Bei Gleichstand bleiben die Punktestände unverändert. Beide Kontrahenten legen die in dieser Aktion eingesetzten Karten auf den Ablagestapel und der Spieler, der nun die höchste offene Karte vor sich hat, ist dran.

Für 1-gegen-1-Konflikte wird dies vereinfacht zu: Höhere Karte gewinnt einen EP, niedrigere verliert einen.
\subsection {Konfliktende}
Nach der dritten Runde wird das Konfliktergebnis in Tabelle \ref {tab:konfliktergebnisse} (Seite \pageref {tab:konfliktergebnisse}) abgelesen. Eine negative Anzahl von EP bedeutet stets, dass die Aktion gescheitert ist. Ein vollkommener Erfolg (drei oder mehr EP) beinhaltet neben dem erreichen des Zieles noch einen "`zusätzlichen Gewinn"', der je nach Situation frei zu gestalten ist. Allgemein ist die Interpretation der Konsequenzen für Sieger stärker von der Situation abhängig und obliegt dem Spieler nach Maßgabe des SL.

Sind die Ergebnisse der Beteiligten nicht unabhängig voneinander zu bewerten (z.B.: Seilziehen), hat der SL die Wahl, ob er für jede Partei die Summe bildet (wenn alle gleichwertig zum Ergebnis beitragen), nur das höchste Ergebnis wertet(wenn alle darauf hinarbeiten, dass einer Erfolg hat) oder was auch immer er für das Logischste hält.

\begin{table}[H]
\caption{Konfliktergebnisse}
\label{tab:konfliktergebnisse}
\begin{tabular}{|l|l|l|}
\hline
EP  & Qualität & Konsequenz \\
\hline
\textless -2 & Vollkommen & Dem Schicksal oder dem Gegner total ausgeliefert \\
-2  & Klar & Schwerer Misserfolg \& einSchränkung (Probeattribut -3) \\
-1 & Knapp & knapper Misserfolg \& leichte einSchränkung (Probeattribut -1) \\
+-0 & Unentschieden & Außer Zeit nichts verloren \\
+1 & Knapp & Teilerfolg \\
+2 & Klar & Ziel erreicht \\
\textgreater +2 & Vollkommen & Erfolg und zusätzlicher Gewinn \\
\hline
\end{tabular}

Speziell nach sozialen oder intellektuellen Proben unterliegt die eventuelle einSchränkung dem Urteil des SL.
\end{table}

\subsection {Passen}
\label {subsect:passen}
Statt eine Karte zu spielen, darf man auch passen. Dies zählt für Reaktionen als farblose 7, eine gegnerische 7 kann damit also abgewehrt werden. Es ist jedoch nicht möglich, selbst zu agieren, wenn man gepasst hat. Auch darf ein KT, der gepasst hat, in dieser Runde keine Karte mehr spielen.
\subsection {einSchränkungen}
einSchränkungen sind die negativen Folgen einer verlorenen Probe. Es handelt sich um Abzüge auf das Probenattribut. Die Angaben in Tabelle \ref {tab:konfliktergebnisse} (Seite \pageref {tab:konfliktergebnisse}) sollten hier nur als Richtwerte verstanden werden und unterliegen stets dem Urteil des SL.

Erhält ein SC eine einSchränkung, sollte der Spieler den Wert, das Attribut das Datum und die Situation kurz vermerken, damit der SL später besser entscheiden kann, wann die Einschränkung wieder "`geheilt"' ist.
