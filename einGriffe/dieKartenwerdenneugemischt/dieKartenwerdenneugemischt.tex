%einBlatt von Tempestrider & eisenherzz: Dieses Werk ist unter einem Creative Commons Attribution-NonCommercial-ShareAlike 3.0 Germany Lizenzvertrag lizenziert. Um die Lizenz anzusehen, gehen Sie bitte zu http://creativecommons.org/licenses/by-nc-sa/3.0/de/ oder schicken Sie einen Brief an Creative Commons, 171 Second Street, Suite 300, San Francisco, California 94105, USA.
\section{Die Karten werden neu gemischt}
Statt auf einen Ablagestapel legen die Spieler ihre Karten nach dem Konflikt auf zwei Stapel. Ist der Nachziehstapel aufgebraucht, werden beide getrennt gemischt, wodurch der Spieler nun über zwei verschiedene Nachziehstapel verfügt. Wenn Karten zu ziehen sind, steht es dem Spieler frei, von welchem Stapel er sie zieht. Erst wenn beide aufgebracuht sind, darf erneut gemischt werden.
\\
Es gibt verschiedene Möglichkeiten, nach welchen Kriterien die Karten auf die beiden Stapel augeteilt werden können:
\begin{itemize}
\item ein Stapel kann für die Karten reserviert werden, die in Probene eingesetzt wurden, der andere für diejenigen, die ungenutzt abgeworfen wurden
\item ein Stapel könnte für die Karten sein, die eine Probe gewonnen haben, der andere für den Rest
\item es kann nach Farben oder nach Bildern und Zahlen unterschieden werden (was den Zufallsfaktor deutlich mindert und die SCs merklich mächtiger macht; insbesondere sinkt die Wahrscheinlichkeit derber Patzer)
\item der SL kann es den Spieler vollkommen freistellen (was die SCs erheblich mächtiger macht)
\end{itemize}
Des weiteren muss der SL (oder die Gruppe) noch festlegen, ob immer alle Karten für eine Probe vom selben Stapel zu ziehen sind oder ob zwischendrin gewechselt werden darf.

