%einBlatt von Tempestrider & eisenherzz: Dieses Werk ist unter einem Creative Commons Attribution-NonCommercial-ShareAlike 3.0 Germany Lizenzvertrag lizenziert. Um die Lizenz anzusehen, gehen Sie bitte zu http://creativecommons.org/licenses/by-nc-sa/3.0/de/ oder schicken Sie einen Brief an Creative Commons, 171 Second Street, Suite 300, San Francisco, California 94105, USA.
\part {einGriffe}
\chapter {einGriffe}

einGriffe sind eng verwandt mit einRichtungen. Auch sie sind Regelmodule, aber sie verändern nicht die Spielwelt oder die Charaktere, sondern die Regeln selbst. Klingt zu abstrakt?\\
(Ja, das hören wir öfter\dots)

Nehmen wir als Beispiel eine Regel, die Spielercharakteren nach einer knapp verlorenen Probe anbietet, eine neue Karte zu ziehen und gegen eine ihrer unterlegenen Karten zu tauschen, wenn sie im Gegenzug nach der Probe einen 2 Punkte Malus auf ihr Attribut hinnehmen.

Diese Regel verändert die Spielwelt nicht. Die Charaktere bekommen zwar eine neue taktische Möglichkeit, aber keine, die man als Teil des Charakters sehen kann. Diese Regel könnte nach der dritten Session in einer Runde probehalber eingeführt und zwei Sessions später bei Nichtgefallen wieder fallen gelassen werden -- das macht sie zu einem einGriff.

