%einBlatt von Tempestrider & eisenherzz: Dieses Werk ist unter einem Creative Commons Attribution-NonCommercial-ShareAlike 3.0 Germany Lizenzvertrag lizenziert. Um die Lizenz anzusehen, gehen Sie bitte zu http://creativecommons.org/licenses/by-nc-sa/3.0/de/ oder schicken Sie einen Brief an Creative Commons, 171 Second Street, Suite 300, San Francisco, California 94105, USA.
\section{Alles auf eine Karte setzen}
Verliert ein Charakter einen Konflikt "`knapp"' und stellt ein Beharren auf dem Konflikt ein interessantes, zusätzliches Risiko dar (SL-Entscheidung), darf er Alles Auf Eine Karte Setzen: er deckt nochmal eine Karte vom Stapel auf und spielt sie anstelle einer der seiner Karten aus dem vorangegangenen Konflikt. Erreicht er so einen 2:1 - Sieg oder ein Unentschieden, prima (aber natürlich darf in einem solchen Fall sein Gegner nun ebenfalls Alles Auf Eine Karte Setzen). Ändert sich jedoch nichts am Ergebnis, so wird der Konflikt automatisch als 3:0 gegen ihn gewertet und die vereinbarte Gefahr tritt ein.
\\
\\
Pro Konflikt kann jeder Charakter nur ein Mal Alles Auf Eine Karte Setzen.
\\
\\
In einem knappen Kampf kann man stets Alles Auf Eine Karte Setzen -- das drohende 3:0 ist Risiko genug\dots
