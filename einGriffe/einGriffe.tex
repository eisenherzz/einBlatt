%einBlatt von Tempestrider & eisenherzz: Dieses Werk ist unter einem Creative Commons Attribution-NonCommercial-ShareAlike 3.0 Germany Lizenzvertrag lizenziert. Um die Lizenz anzusehen, gehen Sie bitte zu http://creativecommons.org/licenses/by-nc-sa/3.0/de/ oder schicken Sie einen Brief an Creative Commons, 171 Second Street, Suite 300, San Francisco, California 94105, USA.
\part {einGriffe}
\chapter {einGriffe}

einGriffe sind eng verwandt mit einRichtungen. Auch sie sind Regelmodule, aber sie verändern nicht die Spielwelt oder die Charaktere, sondern die Regeln selbst. Klingt zu abstrakt?\\
(Ja, das höre ich öfter\dots)

Nehmen wir als Beispiel eine Regel, die Spielercharakteren nach einer knapp verlorenen Probe anbietet, eine neue Karte zu ziehen und gegen eine ihrer unterlegenen Karten zu tauschen, wenn sie im Gegenzug nach der Probe einen 2 Punkte Malus auf ihr Attribut hinnehmen.

Diese Regel verändert die Spielwelt nicht. Die Charaktere bekommen zwar eine neue taktische Möglichkeit, aber keine, die man als Teil des Charakters sehen kann. Diese Regel könnte nach der dritten Session in einer Runde probehalber eingeführt und zwei Sessions später bei Nichtgefallen wieder fallen gelassen werden -- das macht sie zu einem einGriff.

%einBlatt von Tempestrider & eisenherzz: Dieses Werk ist unter einem Creative Commons Attribution-NonCommercial-ShareAlike 3.0 Germany Lizenzvertrag lizenziert. Um die Lizenz anzusehen, gehen Sie bitte zu http://creativecommons.org/licenses/by-nc-sa/3.0/de/ oder schicken Sie einen Brief an Creative Commons, 171 Second Street, Suite 300, San Francisco, California 94105, USA.
\section{Sich in die Karten schauen lassen}
Wer eine Karte offen auslegt (in den Konfliktbereich, nicht auf den Ablagestapel!), darf eine beliebige Anzahl Karten von seiner Hand abwerfen und ebenso viele neue von seinem Nachziehstapel ziehen. Solange der Nachziehstapel nicht leer und noch wenigstens ein Slot frei ist, darf dies wiederholt werden. Alle Karten in der Hand haben darf auf diese Weise nicht herbeigeführt werden.

%einBlatt von Tempestrider & eisenherzz: Dieses Werk ist unter einem Creative Commons Attribution-NonCommercial-ShareAlike 3.0 Germany Lizenzvertrag lizenziert. Um die Lizenz anzusehen, gehen Sie bitte zu http://creativecommons.org/licenses/by-nc-sa/3.0/de/ oder schicken Sie einen Brief an Creative Commons, 171 Second Street, Suite 300, San Francisco, California 94105, USA.
\section{Die Karten zinken}
Hat sich ein SC besonders gut auf einen bevorstehenden Konflikt vorbereitet (SL-Entscheidung), darf er direkt vor dem Konflikt 1 - 5 Karten (wieder SL-Entscheidung) von seinem Nachziehstapel aufnehmen, eine beliebige Anzahl direkt auf seinem Ablagestapel entsorgen und den Rest wieder zurück auf den Nachziehstapel legen.
\\
\textbf{Achtung: sehr mächtig!}


%einBlatt von Tempestrider & eisenherzz: Dieses Werk ist unter einem Creative Commons Attribution-NonCommercial-ShareAlike 3.0 Germany Lizenzvertrag lizenziert. Um die Lizenz anzusehen, gehen Sie bitte zu http://creativecommons.org/licenses/by-nc-sa/3.0/de/ oder schicken Sie einen Brief an Creative Commons, 171 Second Street, Suite 300, San Francisco, California 94105, USA.
\section{Nachkarten}
Geht ein Konflikt unentschieden aus, hat jeder der Beteiligten das Recht, ein Mal nachzukarten, indem er offen eine weitere Karte ausspielt. Kann ein Gegner keine zumindest gleichwertige Karte spielen, gilt er als knapp besiegt. Kann er eine genau gleichwertige oder höhere Karte dagegen spielen, gilt nach wie vor das Unentschieden, aber er darf nun seinerseits nachkarten (sofern er das in diesem Konflikt nicht schon getan hat).

%einBlatt von Tempestrider & eisenherzz: Dieses Werk ist unter einem Creative Commons Attribution-NonCommercial-ShareAlike 3.0 Germany Lizenzvertrag lizenziert. Um die Lizenz anzusehen, gehen Sie bitte zu http://creativecommons.org/licenses/by-nc-sa/3.0/de/ oder schicken Sie einen Brief an Creative Commons, 171 Second Street, Suite 300, San Francisco, California 94105, USA.
\section{Die Karten werden neu gemischt}
Statt auf einen Ablagestapel legen die Spieler ihre Karten nach dem Konflikt auf zwei Stapel. Ist der Nachziehstapel aufgebraucht, werden beide getrennt gemischt, wodurch der Spieler nun über zwei verschiedene Nachziehstapel verfügt. Wenn Karten zu ziehen sind, steht es dem Spieler frei, von welchem Stapel er sie zieht. Erst wenn beide aufgebracuht sind, darf erneut gemischt werden.
\\
Es gibt verschiedene Möglichkeiten, nach welchen Kriterien die Karten auf die beiden Stapel augeteilt werden können:
\begin{itemize}
\item ein Stapel kann für die Karten reserviert werden, die in Probene eingesetzt wurden, der andere für diejenigen, die ungenutzt abgeworfen wurden
\item ein Stapel könnte für die Karten sein, die eine Probe gewonnen haben, der andere für den Rest
\item es kann nach Farben oder nach Bildern und Zahlen unterschieden werden (was den Zufallsfaktor deutlich mindert und die SCs merklich mächtiger macht; insbesondere sinkt die Wahrscheinlichkeit derber Patzer)
\item der SL kann es den Spieler vollkommen freistellen (was die SCs erheblich mächtiger macht)
\end{itemize}
Des weiteren muss der SL (oder die Gruppe) noch festlegen, ob immer alle Karten für eine Probe vom selben Stapel zu ziehen sind oder ob zwischendrin gewechselt werden darf.


%einBlatt von Tempestrider & eisenherzz: Dieses Werk ist unter einem Creative Commons Attribution-NonCommercial-ShareAlike 3.0 Germany Lizenzvertrag lizenziert. Um die Lizenz anzusehen, gehen Sie bitte zu http://creativecommons.org/licenses/by-nc-sa/3.0/de/ oder schicken Sie einen Brief an Creative Commons, 171 Second Street, Suite 300, San Francisco, California 94105, USA.
\section{Die Karten auf den Tisch legen}
Vor dem Auslegen der Karten kann jeder Spieler statt Karten auszulegen alle seine Handkarten offen abwerfen und stattdessen eine neue Hand ziehen, die jedoch um 1 Karte kleiner ist als die zuvor abgelegte. Dies kann pro Charakter und Konflikt nur ein Mal gemacht werden.

%einBlatt von Tempestrider & eisenherzz: Dieses Werk ist unter einem Creative Commons Attribution-NonCommercial-ShareAlike 3.0 Germany Lizenzvertrag lizenziert. Um die Lizenz anzusehen, gehen Sie bitte zu http://creativecommons.org/licenses/by-nc-sa/3.0/de/ oder schicken Sie einen Brief an Creative Commons, 171 Second Street, Suite 300, San Francisco, California 94105, USA.
\section{Ein Ass im Ärmel haben}
Hat ein Spieler in einem Konflikt mehr als die benötigten drei Karten auf der Hand, kann er eine als Ass im Ärmel benutzen. Dazu legt er sie verdeckt vor sich ab. In einem späteren Konflikt kann er eine bereits aufgedeckte Karte durch das Ass im Ärmel ersetzen. Man darf nie mehr als ein Ass im Ärmel haben. Das Ass verbleibt im Ärmel, bis es in einem Konflikt ausgespielt wird.

%einBlatt von Tempestrider & eisenherzz: Dieses Werk ist unter einem Creative Commons Attribution-NonCommercial-ShareAlike 3.0 Germany Lizenzvertrag lizenziert. Um die Lizenz anzusehen, gehen Sie bitte zu http://creativecommons.org/licenses/by-nc-sa/3.0/de/ oder schicken Sie einen Brief an Creative Commons, 171 Second Street, Suite 300, San Francisco, California 94105, USA.
\section{Alle Karten in der Hand haben}
Hat ein Spieler weniger Karten auf seinem Nachziehstapel als er ziehen darf, so zieht er nur die vorhandene Anzahl. Er darf jedoch im bevorstehenden Konflikt einen Slot frei lassen, der automatisch als König der entsprechenden Attributsfarbe gewertet wird. Muss der Spieler einen weiteren Slot frei lassen (weil er nur eine Karte ziehen konnte), zählt dieser als farblose "`7"'.

%einBlatt von Tempestrider & eisenherzz: Dieses Werk ist unter einem Creative Commons Attribution-NonCommercial-ShareAlike 3.0 Germany Lizenzvertrag lizenziert. Um die Lizenz anzusehen, gehen Sie bitte zu http://creativecommons.org/licenses/by-nc-sa/3.0/de/ oder schicken Sie einen Brief an Creative Commons, 171 Second Street, Suite 300, San Francisco, California 94105, USA.
\section{Alles auf eine Karte setzen}
Verliert ein Charakter einen Konflikt "`knapp"' und stellt ein Beharren auf dem Konflikt ein interessantes, zusätzliches Risiko dar (SL-Entscheidung), darf er Alles Auf Eine Karte Setzen: er deckt nochmal eine Karte vom Stapel auf und spielt sie anstelle einer der seiner Karten aus dem vorangegangenen Konflikt. Erreicht er so einen 2:1 - Sieg oder ein Unentschieden, prima (aber natürlich darf in einem solchen Fall sein Gegner nun ebenfalls Alles Auf Eine Karte Setzen). Ändert sich jedoch nichts am Ergebnis, so wird der Konflikt automatisch als 3:0 gegen ihn gewertet und die vereinbarte Gefahr tritt ein.
\\
\\
Pro Konflikt kann jeder Charakter nur ein Mal Alles Auf Eine Karte Setzen.
\\
\\
In einem knappen Kampf kann man stets Alles Auf Eine Karte Setzen -- das drohende 3:0 ist Risiko genug\dots

%einBlatt von Tempestrider & eisenherzz: Dieses Werk ist unter einem Creative Commons Attribution-NonCommercial-ShareAlike 3.0 Germany Lizenzvertrag lizenziert. Um die Lizenz anzusehen, gehen Sie bitte zu http://creativecommons.org/licenses/by-nc-sa/3.0/de/ oder schicken Sie einen Brief an Creative Commons, 171 Second Street, Suite 300, San Francisco, California 94105, USA.
\section{Mit offenen Karten spielen}
Hat eine Konfliktpartei einen taktischen Nachteil (Hinterhalt, Drogeneinfluss, \dots), so kann der SL dies berücksichtigen, indem er sie mit offenen Karten spielen lässt -- je nach Umfang des Nachteils müssen 1, 2 oder sogar alle drei Karten offen ausgelegt werden, bevor der Gegner seine Karten spielt.

%einBlatt von Tempestrider & eisenherzz: Dieses Werk ist unter einem Creative Commons Attribution-NonCommercial-ShareAlike 3.0 Germany Lizenzvertrag lizenziert. Um die Lizenz anzusehen, gehen Sie bitte zu http://creativecommons.org/licenses/by-nc-sa/3.0/de/ oder schicken Sie einen Brief an Creative Commons, 171 Second Street, Suite 300, San Francisco, California 94105, USA.
\section{Gnade der Götter / Zorn der Götter (HÖL)}

Dieser einGriff ist eine Adaptation des gleichnamigen Regelpaares aus dem krankhaft genialen, leider seit Jahren vergriffenen Satirerollenspiel "`HÖL – Human Occupied Landfill"' aus White Wolfs Black Dog Game Factory. Schon Anfang der 90er strikt auf vier stimmungsvolle Attribute (Meat, Feet, Nuts und Greymatta) reduziert wirkt HÖL heute quasi wie ein frühgeborenes einBlatt. Während man das Grundregelwerk liest, wirkt die Realität wie ein einziger, langer Drogenentzug. Ehre, wem Ehre gebührt.

\subsection{Sessionvorbereitung}
Am Anfang jeder Session zieht der SL verdeckt zwei "`Zorn der Götter"'-Karten, (ZdGK) schaut sie an, merkt sie sich und legt sie beiseite.

\subsection{Im Spiel}

Verliert ein SC einen Konflikt, darf er die "`Gnade der Götter"' (des Schicksals, Saurons, seiner Mutter, die von Herrn Kaiser von der Habsburg-Mülleimer, etc.) erflehen. Daraufhin zieht er drei Karten (GdGK), mit denen er eine beliebige Auswahl seiner zuvor gespielten Karten ersetzen und den verlorenen Konflikt so vielleicht doch noch gewinnen kann.

Währenddessen vergleicht der SL diese drei Karten mit seinen ZdGK. Findet er keine Übereinstimmung, zieht er eine weitere ZdGK und das Spiel geht mit dem von den Göttern gesegneten Konfliktausgang weiter. Findet er eine Karte, die sowohl unter den GdGK als auch unter den ZdGK ist, befällt ein unwahrscheinliches Unglück den SC, der die Gnade so überstrapaziert hat -- er verliert den Konflikt vollkommen und muss selbst ein sowohl schmerzhaftes als auch für alle Zuschauer lustiges Scheitern erzählen.

Findet der SL mehr als eine Übereinstimmung, trifft der ZdG die ganze, gierige Gruppe\dots

In beiden Fällen mischt der SL alle ZdGK unter sein Deck und zieht zwei neue.

Wenn der SL seinen Ablagestapel neu mischt, mischt er die ZdGK ebenfalls mit, zieht aber danach die halbe Anzahl (aufgerundet) neu.

\subsection{Und was macht man damit?}

Diese Option eignet sich speziell für "`Fun-Runden"', in denen kreative Katastrophen erwünschte Spaßfaktoren sind -- denn genau das liefert dieser Mechanismus.

In "`ernsthafteren"' Runden werden die Spieler schnell dahinter kommen, dass das Risiko im Vergleich zum zu erwartenden Gewinn meist viel zu hoch ist und deshalb nur in den allerschlimmsten Momenten die "`Gnade der Götter"' erbitten. Doch auch hier kann die Regel mit ein paar Anpassungen helfen, die Dramatik zu steigern:
\begin{enumerate}
\item Die Götter und ihre Macht müssen klarer umrissen werden, z.B.:
\begin{itemize}
\item Fantasy: Götter; mächtige Geister; Schutzengel; Magierkönige
\item Modern: politische Widerstandskämpfer; die Matrix, eine geheime Militäreinheit
\item Mystery: Geister; Wesen aus einer anderen Dimension; ein fremdartiges Artefakt
\item SF: mächtige Aliens; eine geheimnisvolle, unzuverlässige Technologie
\end{itemize}
\item Der ZdG muss klar definiert und stark auf das jeweilige Spiel abgestimmt werden, z.B.:
\begin{itemize}
\item Mystery: Für eine zufällige Anzahl von Szenen wird der SC von einer fremden Intelligenz übernommen (Der SL gibt dem Spieler ein paar Anweisungen auf einem Zettel)
\item Modern mit Widerstandskämpfern: die befreundeten Rebellen werden beim Versuch zu helfen von der Geheimpolizei erwischt
\item SF mit geheimnisvoller Technologie: der gewünschte Effekt tritt viel zu groß / klein / am falschen Ziel auf.
\end{itemize}
\item Der SL muss für eine Vielzahl existenzieller Konflikte (also solcher, die die Spieler um fast jeden Preis gewinnen wollen) sorgen.
\item Die "`Gnade der Götter"' wird nicht nach dem Konflikt, sondern direkt nach dem Kartenziehen erbeten.
\item Löst die Gnade kleinen Zorn aus, wird nur eine neue ZdGK gezogen.
\item Beim Neumischen zieht der SL nur eine Zorn-Karte weniger als er vor dem Mischen hatte.
\end{enumerate}


