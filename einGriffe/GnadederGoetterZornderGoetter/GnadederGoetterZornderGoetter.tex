%einBlatt von Tempestrider & eisenherzz: Dieses Werk ist unter einem Creative Commons Attribution-NonCommercial-ShareAlike 3.0 Germany Lizenzvertrag lizenziert. Um die Lizenz anzusehen, gehen Sie bitte zu http://creativecommons.org/licenses/by-nc-sa/3.0/de/ oder schicken Sie einen Brief an Creative Commons, 171 Second Street, Suite 300, San Francisco, California 94105, USA.
\section{Gnade der Götter / Zorn der Götter (HÖL)}

Dieser einGriff ist eine Adaptation des gleichnamigen Regelpaares aus dem krankhaft genialen, leider seit Jahren vergriffenen Satirerollenspiel "`HÖL – Human Occupied Landfill"' aus White Wolfs Black Dog Game Factory. Schon Anfang der 90er strikt auf vier stimmungsvolle Attribute (Meat, Feet, Nuts und Greymatta) reduziert wirkt HÖL heute quasi wie ein frühgeborenes einBlatt. Während man das Grundregelwerk liest, wirkt die Realität wie ein einziger, langer Drogenentzug. Ehre, wem Ehre gebührt.

\subsubsection{Sessionvorbereitung}
Am Anfang jeder Session zieht der SL verdeckt zwei "`Zorn der Götter"'-Karten, (ZdGK) schaut sie an, merkt sie sich und legt sie beiseite.

\subsubsection{Im Spiel}

Verliert ein SC einen Konflikt, darf er die "`Gnade der Götter"' (des Schicksals, Saurons, seiner Mutter, die von Herrn Kaiser von der Habsburg-Mülleimer, etc.) erflehen. Daraufhin zieht er drei Karten (GdGK), mit denen er eine beliebige Auswahl seiner zuvor gespielten Karten ersetzen und den verlorenen Konflikt so vielleicht doch noch gewinnen kann.

Währenddessen vergleicht der SL diese drei Karten mit seinen ZdGK. Findet er keine Übereinstimmung, zieht er eine weitere ZdGK und das Spiel geht mit dem von den Göttern gesegneten Konfliktausgang weiter. Findet er eine Karte, die sowohl unter den GdGK als auch unter den ZdGK ist, befällt ein unwahrscheinliches Unglück den SC, der die Gnade so überstrapaziert hat -- er verliert den Konflikt vollkommen und muss selbst ein sowohl schmerzhaftes als auch für alle Zuschauer lustiges Scheitern erzählen.

Findet der SL mehr als eine Übereinstimmung, trifft der ZdG die ganze, gierige Gruppe\dots

In beiden Fällen mischt der SL alle ZdGK unter sein Deck und zieht zwei neue.

Wenn der SL seinen Ablagestapel neu mischt, mischt er die ZdGK ebenfalls mit, zieht aber danach die halbe Anzahl (aufgerundet) neu.

\subsubsection{Und was macht man damit?}

Diese Option eignet sich speziell für "`Fun-Runden"', in denen kreative Katastrophen erwünschte Spaßfaktoren sind -- denn genau das liefert dieser Mechanismus.

In "`ernsthafteren"' Runden werden die Spieler schnell dahinter kommen, dass das Risiko im Vergleich zum zu erwartenden Gewinn meist viel zu hoch ist und deshalb nur in den allerschlimmsten Momenten die "`Gnade der Götter"' erbitten. Doch auch hier kann die Regel mit ein paar Anpassungen helfen, die Dramatik zu steigern:
\begin{enumerate}
\item Die Götter und ihre Macht müssen klarer umrissen werden, z.B.:
\begin{itemize}
\item Fantasy: Götter; mächtige Geister; Schutzengel; Magierkönige
\item Modern: politische Widerstandskämpfer; die Matrix, eine geheime Militäreinheit
\item Mystery: Geister; Wesen aus einer anderen Dimension; ein fremdartiges Artefakt
\item SF: mächtige Aliens; eine geheimnisvolle, unzuverlässige Technologie
\end{itemize}
\item Der ZdG muss klar definiert und stark auf das jeweilige Spiel abgestimmt werden, z.B.:
\begin{itemize}
\item Mystery: Für eine zufällige Anzahl von Szenen wird der SC von einer fremden Intelligenz übernommen (Der SL gibt dem Spieler ein paar Anweisungen auf einem Zettel)
\item Modern mit Widerstandskämpfern: die befreundeten Rebellen werden beim Versuch zu helfen von der Geheimpolizei erwischt
\item SF mit geheimnisvoller Technologie: der gewünschte Effekt tritt viel zu groß / klein / am falschen Ziel auf.
\end{itemize}
\item Der SL muss für eine Vielzahl existenzieller Konflikte (also solcher, die die Spieler um fast jeden Preis gewinnen wollen) sorgen.
\item Die "`Gnade der Götter"' wird nicht nach dem Konflikt, sondern direkt nach dem Kartenziehen erbeten.
\item Löst die Gnade kleinen Zorn aus, wird nur eine neue ZdGK gezogen.
\item Beim Neumischen zieht der SL nur eine Zorn-Karte weniger als er vor dem Mischen hatte.
\end{enumerate}

