%einBlatt von Tempestrider & eisenherzz: Dieses Werk ist unter einem Creative Commons Attribution-NonCommercial-ShareAlike 3.0 Germany Lizenzvertrag lizenziert. Um die Lizenz anzusehen, gehen Sie bitte zu http://creativecommons.org/licenses/by-nc-sa/3.0/de/ oder schicken Sie einen Brief an Creative Commons, 171 Second Street, Suite 300, San Francisco, California 94105, USA.
\part {Nur Der Anfang}
\chapter {Nur der Anfang}

Falls Du (wie es die meisten Rollenspieler zu tun pflegen) dieses Heftchen schon mal vor dem Lesen durchgeblättert hast, wird Dich wahrscheinlich eine Frage beschlichen haben:

\section {Was ist dieses "`einBlatt"'?}
Es sieht ein bisschen so aus wie ein Rollenspiel (sag bitte nicht "`Amateur-Rollenspiel"' -- wir bevorzugen die Bezeichnung "`Indie-RPG"'), aber irgendwie auch nicht: Nichts ist so angeordnet, wie Du es gewohnt bist, am Anfang diese Grundregeln, das Setting erst kurz vor dem Ende, dazwischen all das, was Du normalerweise in den Grundregeln erwartet hättest und ganz zum Schluss wieder irgendwelche Regeln. Andererseits ist aber alles da was man braucht, und wenn man will, kann man es wahrscheinlich auch spielen.

\section {Ist einBlatt also ein (besonders schlecht sortiertes) Rollenspiel?}
Nun, das kann es sein. Aber eigentlich ist es viel mehr. Wie Du beim Lesen feststellen wirst, besteht einBlatt nicht aus vielen komplex miteinander verwobenen Regeln. Stattdessen gibt es einerseits einen leichten, offenen Regelkern ("`einRegelBlatt"') und andererseits eine Vielzahl handlicher Regelmodule, mit denen einRegelBlatt zu einem kompletten Spiel -- wie dem  auf Seite \pageref {sect:einVerlies}, Kapitel \ref {sect:einVerlies}, beschriebenen "`einVerlies -- Im Dungeon gibt es keine Maniküre"' -- ausgebaut werden kann. Insofern gehören eine ganze Reihe Rollenspiele zu einBlatt. 

\section {Also ist einBlatt ein generisches Rollenspiel?}
Viele Spiele haben schon Regeln und Setting voneinander getrennt (seit GURPS nennt man das "`generisch"', und das klingt nicht zufällig wie der kleine Bruder von "`beliebig"'). einBlatt geht noch einen großen Schritt weiter: es trennt die Regeln voneinander. Du entscheidest nicht nur, ob Du Fantasy, Horror, Space Opera oder Mystery spielen willst, sondern auch ob Du auf ein einfaches, ein taktisches, ein cineastisches oder ein brutales Kampfsystem setzt, ob Du ein spruchbasiertes, ein freies oder gar kein Magiesystem bevorzugst, ob Deine Spieler miteinander oder gegeneinander spielen, ob sie nur die Handlungen ihrer Charaktere beschreiben dürfen oder mit Dir zusammen die Spielwelt erschaffen sollen, ob Du Mechanismen willst, die bestimmtes Charakterverhalten belohnen und / oder bestrafen, ob Du klassische Proben auf Handlungen willst oder lieber diese neumodischen Konfliktproben -- einBlatt schreibt es Dir nicht vor, sondern ermöglicht es Dir. So befreit es Dich einerseits von den Vorgaben "`generischer"' Systeme, gibt dir aber am Ende ein Spiel, das nicht "`halbwegs auf Deine Idee angepasst"' ist, sondern ganz genau!
\\
einBlatt ist kein Rollenspiel, generisch oder anderweitig -- Rollenspiel ist, was Du daraus machst.

\section{Warum setzt einBlatt auf Karten statt auf Würfel?}
\begin{enumerate}
\item Mit Karten kannst Du (als Spieleentwickler oder Spielleiter) selber kontrollieren, wie zufällig die Ergebnisse wirklich sein sollen:
\begin{itemize}
\item wenn Du nach jeder Probe neu mischst hast Du praktisch dieselbe Zufallsverteilung wie beim Würfel
\item wenn Du den Stapel komplett durchspielen lässt bevor neu gemischt wird, hast Du eine relativ gleichmäßige Verteilung von Glück und Pech
\item wenn Du alle Spieler vom selben Stapel ziehen lässt, ist das Glück des Einen das Pech der Anderen wenn Du den Zeitpunkt des Mischens von Spielereignissen abhängig machst (z.B. eine Nacht Schlaf in einem actiongeladenen Zombie-Rollenspiel), erzeugst Du weitere taktische Optionen
\end{itemize}      
Du kannst alles machen, was Du mit Würfeln kannst, musst aber nicht, sprich: Du bist einfach flexibler als mit Würfeln.
\item Karten haben nicht nur einen Zahlenwert, sondern auch eine Farbe. Natürlich kann man das auch wie einen 32-seitigen Würfel sehen oder umgekehrt mit bunten Würfeln arbeiten. Mit Karten ist das aber deutlich einfacher und intuitiver.
\item Du kannst dem Spieler eine Menge weiterer taktischer Optionen an die Hand geben -- alleine schon die Möglichkeiten einer Kartenhand, aus der der Spieler selbst wählt, welche Karten er wann spielt und welche er zurückhält. Auch das ist mit Würfeln nicht komplett unmöglich, aber doch reichlich kompliziert.
\end{enumerate}
Insbesondere im Hinblick auf die Möglichkeiten, die damit für die Entwicklung von Erweiterungen geschaffen werden, erscheinen uns Karten als die sinnvollere Alternative.

\section {Mit diesem halben Dutzend Regelmodulen kommt man aber nicht weit!}
Stimmt. Deshalb ist einBlatt euch ein Internet-Projekt. Derzeit wird der Quelltext auf \url{https://github.com/eisenherzz/einBlatt} gehostet. Benutze das Ticketsystem bei github \url{https://github.com/eisenherzz/einBlatt/issues} . Dort ist eine öffentliche Diskussion über Fehler und Vorschläge möglich. Hier können registrierte User dabei helfen, Module zu verbessern bis sie fertige einBlätter sind oder selbst neue Module beisteuern. Alternativ nutze das Kontaktformular auf \url{http://www.eisenherzz.de} (Kategorie: zu einBlatt beitragen). Dort bekommt man auch immer das aktuelle PDF.
\\
Noch nie war es so einfach, selbst ein Rollenspiel zu entwickeln.

\section {Und was ist dann dieses Heftchen?}
Dieses Heftchen ist nur ein kleiner Ausschnitt von einBlatt -- die Regeln für das erste komplette einBlatt-Rollenspiel "`einVerlies -- Im Dungeon gibt es keine Maniküre"'. So wie Du es gerade in der Hand hältst ist es kostenlos und darf unter der Creative Commons Lizenz frei verteilt werden -- so wie alle Bestandteile von einBlatt. 

\section {Und jetzt?}
Viel Spass beim Lesen, Spielen und hoffentlich mitmachen! Über Fragen, Anregungen und jede andere konstruktive Form der (An-)Teilnahme freuen wir uns unter:

\Letter einblatt@eisenherzz.de
\\
\\
Tempestrider \& eisenherzz
