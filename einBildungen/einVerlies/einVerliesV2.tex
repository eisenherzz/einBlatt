%einBlatt von Tempestrider & eisenherzz: Dieses Werk ist unter einem Creative Commons Attribution-NonCommercial-ShareAlike 3.0 Germany Lizenzvertrag lizenziert. Um die Lizenz anzusehen, gehen Sie bitte zu http://creativecommons.org/licenses/by-nc-sa/3.0/de/ oder schicken Sie einen Brief an Creative Commons, 171 Second Street, Suite 300, San Francisco, California 94105, USA.

\section{einVerlies -- Im Dungeon gibt es keine Maniküre}
\label{sect:einVerlies}
Tief unter den Mauern der von einem bösen Herrscher beherrschten Festung Karamel, in einem absurd verworrenen, von Sklavenhänden in jahrhundertelanger Schwerstarbeit in den Fels getriebenen Tunnelkomplex für den es keine praktische Verwendung gibt, liegt ein gar wertvoller, von 1W6 Drachen bewachter Schatz. Die Helden (andere SCs existieren in einVerlies nicht!!!) wurden ausgesandt, diesen Schatz den Klauen des bösen (verdammt eklig BÖSEN!!!) Herrschers zu entreißen, da dies die einzige Möglichkeit ist, die gute (gutaussehende) Jungfrau (ehrlich!) des Königreiches Ril'ngirna'Krum-Lma'A von ihrem ewigen Mundgeruch zu erlösen, auf dass der heroischste unter den Helden sie Freien und alsbald König anstelle des guten, doch altersschwachen Königs werde.

Die Prinzessin ist sehr flexibel, was im Interesse aller Heldinnen zu übersetzen ist mit bi- oder transsexuell -- aber wie es sich für eine gute Jungfrau geziemt, ist sie absolut monogam.

Es kann also nur einen neuen König geben.

 

\subsection{Regeln}

\subsubsection{Charaktererschaffung}


Es wird ein fünftes Attribut benötigt: Heldenhafte Schönheit. Auf dieses werden im Verlauf des Spiels keinerlei Proben abgelegt -- außer bei der Übergabe des Schatzes an den altersschwachen König.

Für Erfahrungspunkte setzt einVerlies die Erfahrungspunkte für Powergamer ein (siehe Kapitel \ref {sect:ErfahrungspunktefuerPowergamer}, Seite \pageref {sect:ErfahrungspunktefuerPowergamer}).


 
\subsubsection{Heldenhafte Schönheit}

die Schönheit eines Helden, sein makelloses Antlitz und sein narbenfreier Körper, für jeden ahnungslosen Betrachter der untrügliche Beleg seiner Unbesiegbarkeit, ist die nutzloseste Eigenschaft, die man im Kampf gegen eine Hundertschaft von Grottenolmen haben kann. Gleichzeitig ist es die eine Eigenschaft, mit der man sich aus dieser miesen Existenz als Kammerjäger des Hochadels freilächeln und selbst König anstelle des Königs werden kann. Ergo sollte jeder Spieler sowohl in der Charaktererschaffung als auch im weiteren Verlauf des Spiels größten Wert auf die Heldenhafte Schönheit seines SCs legen.

Leider tun seine Gegner dies nicht -- egal, ob es gemeine Wald- und Wiesenorks, Schwärme von Fleischermotten oder die Hortdrachen selbst sind, alle zielen nur auf das Gesicht eines jeden Helden. Wird also ein Kampf oder eine Probe gegen eine Falle nicht mit 3 Punkten Differenz gewonnen, sinkt die Schönheit des SC um 1 -- verliert er gar, so sinkt sie um 2.

 
\subsubsection{Das Endspiel}

Nachdem der Schatz geborgen ist und die Helden zum König und seiner ach so holden Tochter zurückgekehrt sind, erhalten sie nicht nur die Belohnung für ihre Leistungen in Form von Erfahrungspunkten, Titeln und / oder Reichtümern, sondern sie wetteifern auch um die Gunst der infantile Triebfantasien auslösenden Prinzessin. Dies geschieht, indem sie gegeneinander einen Konflikt auf Heldenhafte Schönheit ausfechten. Solange es mehrere Sieger gibt, flirten diese weiter, bis ein einziger Sieger feststeht. Dieser hat das Herz der Prinzessin gewonnen, muss sie heiraten und wird umgehend König (was seine Karriere als Held jäh beendet). Der Spieler ist angehalten, noch rasch ein paar neue Gesetzte zu erlassen, ehe er sich einen neuen SC baut.

P.S.: Natürlich ist dieser König der nächste Auftraggeber der Gruppe -- und natürlich wird er weiterhin von seinem Spieler dargestellt\dots

 
\subsubsection{Über Verliese}

Ein Verlies hat für gewöhnlich zehn Ebenen, an deren Ende jeweils ein Boss-Gegner zu besiegen ist. Dieser verfügt über eine Kampffertigkeit in Höhe des jeweiligen Levels+2 (Level1: 3 Karten, Level 10: 12 Karten). Nach Belieben kann der SL auf einem höheren Level beginnen, auf einem niedrigeren aufhören, etc., aber die Steigerung muss erhalten bleiben.

Auf jeder zweiten Ebene darf der Boss-Gegner erst angegriffen werden, nachdem der Rätselmeister\texttrademark, zu erkennen an seinem stets über ihm schwebenden, leicht leuchtendem Fragezeichen, besucht und die von ihm gestellte, zunehmend sinnfreie Aufgabe gelöst worden ist.

Verliese werden nach jedem Besuch von Helden um dekoriert. Einerseits tuen Monster dies für ihr Leben gern, schließlich sind sie alle Ausgebildete Innenarchitekten/Raumausstatter, andererseits wäre es für Helden auch sicher eintönig immer das selbe Verlies zu durchkämmen.

Alles in allem sollte man möglichst viele Klischees bedienen bzw. die ewig währenden Fragen der Verliesbaukunst thematisieren.

Bei weitem nicht vollständige Liste:
\begin{itemize}
\item warum werden Türen mit seltsamen Rätseln anstelle mit einem anständigen Schloss versehen
\item wovon ernähren sich die gar schrecklichen Monster wenn gerade keine Helden da sind
\item verstehen sich die Monster gar untereinander, gehen sie zusammen Kegeln und haben sie einen gemeinsamen Schlafraum
\item ein ewiges Geheimnis mag auch die Nachwuchsfrage bei Verliesbewohnern sein
\item \dots
\end{itemize}

 


